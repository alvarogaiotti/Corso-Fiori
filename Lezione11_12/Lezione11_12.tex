\documentclass[../corsofiori.tex]{subfiles}
\setcounter{regolactr}{39}
\setcounter{chapter}{10}
%\externaldocument[VIII-]{../lezione8/Lezione8}
\setdefaults{colors=4A}

\begin{document}
\chapter{L’intervento}
Si definisce intervento la prima dichiarazione diversa da Passo effettuata da un
Giocatore appartenente alla linea opposta all’Apertore. Un intervento è una libera
iniziativa, sta al giocatore decidere se gli convenga o meno entrare in dichiarazione.

\section{L’ intervento di Contro}

Il Contro esprime il desiderio di aggiudicarsi un contratto sulla propria linea,
giocando in un colore diverso da quello dell’Apertore. La mano perfetta per il Contro
presenta tutti e tre i colori giocabili e la corta in corrispondenza del seme avversario;
ad esempio, su apertura di 1\Di, si dice Contro con:

\smallskip
\hand{KJ98} {AQ74} {5} {K983}
\smallskip

\begin{regola}{L'intervento di Contro}
L’intervento di Contro, in seconda come in quarta posizione,
chiede al compagno di scegliere un colore
e dichiararlo al livello cui pensa di poter giocare.
\end{regola}

Il Contro garantisce o una mano di 12 -16 punti con “la giocabilità” dei Nobili non
dichiarati, o una mano di 17+; in questo caso non ci sono vincoli distribuzionali.

\begin{regola}{Le mani con cui contrare}
Quando si è nella fascia 12-16:
\begin{itemize}
\item Il Contro su 1\Cl o 1\Di garantisce 4 cuori e 4 picche (o 4+3, mai 4+2).
\item Il Contro su 1\He garantisce 4 picche.
\item Il Contro su 1\Sp garantisce 4 cuori.
\end{itemize}
\end{regola}

Il compagno sarà portato a scegliere il maggiore implicitamente promesso, se vi
possiede 4 carte. Ma se non le ha, sceglierà un colore minore: pertanto quando si
interviene con il Contro si deve poter tollerare anche la scelta di un colore minore
(almeno 3 carte, eccezionalmente due, mai una o zero!).

\subsubsection{Esempi}

\newgame\setdefaults{bidfirst=N}
\dealer[\east]\vulner[\eastwest]
%\hand!-{Q952}{K32}{765}{AQ7}
\hand!-{KJ98} {AQ74} {K87} {82}
\begin{bidding}-
& & 1C & ? \\
\end{bidding}
\expertquiz[Sud ha aperto 1\Cl, tocca a Ovest\\]{%
Contro}

Contro; i requisiti sono perfetti: entrambe le quarte maggiori, eventuale tolleranza
delle Quadri, e punteggio sufficiente.

\newgame\setdefaults{bidfirst=N}
\dealer[\east]\vulner[\eastwest]
%\hand!-{Q952}{K32}{765}{AQ7}
\hand!- {9854} {K4} {A765} {AJ3}
\begin{bidding}-
& & 1C & ? \\
\end{bidding}
\expertquiz[]{%
Passo}

Passo; i punti basterebbero sì per intervenire, ma mancano le Cuori. Il compagno, a
cui si sta chiedendo di scegliere un colore (con l’implicita promessa di portargli fit) si
sentirebbe tradito.

Quando il Contro viene dato dal quarto di mano, dopo che gli avversari hanno detto
due colori, la forza onori promessa è sempre 12+, e devono esserci 4 carte in
entrambi i colori che rimangono (quelli su cui si orienterà il compagno):
Sud ha aperto 1\Di, passo di Ovest, Nord ha detto 1\Sp e Est ha:

\newgame\setdefaults{bidfirst=N}
\dealer[\east]\vulner[\eastwest]
%\hand!-{Q952}{K32}{765}{AQ7}
\hand!- {54} {KQ86} {A95} {KQ54}
\begin{bidding}-
& & 1D & p \\
1S & ?\\
\end{bidding}
\expertquiz[]{%
Contro}

Contro: sta chiedendo a Ovest di scegliere tra Fiori e Cuori, e garantisce fit in
entrambi i colori.


\subsection{Come deve comportarsi il compagno}
Il compagno del contrante sceglie un colore perché è obbligato a farlo;
il punteggio minimo con cui fa una dichiarazione al minimo livello è zero.

\paragraph{Esempi:}

\newgame\setdefaults{bidfirst=N}
\dealer[\east]\vulner[\eastwest]
%\hand!-{Q952}{K32}{765}{AQ7}
\hand!-{54}  {J986}  {952}  {Q532}
\begin{bidding}-
& & 1D & X \\
p & ?\\
\end{bidding}
\expertquiz[]{%
1\He}

1\He; anche con una mano quasi nulla è opportuno fare una scelta, la più bassa
possibile, ma non passare: il Passo farebbe diventare definitivo (a discrezione
dell’Apertore, cui spetta l’ultima dichiarazione) il contratto di “ 1\Di contrato”, che si
risolverebbe in un disastro ben peggiore: con molta probabilità verrebbe mantenuto,
e ogni presa fatta in più costerebbe una cifra (100 l’una in Prima, 200 in Seconda!)

\newgame\setdefaults{bidfirst=N}
\dealer[\east]\vulner[\eastwest]
%\hand!-{Q952}{K32}{765}{AQ7}
\hand!-{KJ954}  {A76}  {95}  {K64}
\begin{bidding}-
& & 1D & X \\
p & ?\\
\end{bidding}
\expertquiz[]{%
4\Sp}

4\Sp; fate la dichiarazione che fareste se poteste vedere le carte del compagno: una
mano di almeno 12 punti, con “almeno la 4-3” nei colori nobili.
Ricordate che avete una pallottola sola, quello che dichiarate sul contro non è
l’inizio di un dialogo (come quando il compagno apre e voi rispondete con un colore
nuovo) ma la fine: chi ha contrato non parlerà più. Ed è quello che vi augurate,
quando avete carte come nell’esempio precedente. Morale: se pensate di poter fare
una manche, dichiaratela o resterete al palo.

\newgame\setdefaults{bidfirst=N}
\dealer[\east]\vulner[\eastwest]
%\hand!-{Q952}{K32}{765}{AQ7}
\hand!-{K4}  {Q86}  {AJT}  {K954}
\begin{bidding}-
& & 1D & X \\
p & ?\\
\end{bidding}
\expertquiz[]{%
3\SA}

3\SA: il punteggio è sufficiente, il fit a Cuori o Picche è escluso, il fermo a Quadri c’è:
decisione facile.


\newgame\setdefaults{bidfirst=N}
\dealer[\east]\vulner[\eastwest]
%\hand!-{Q952}{K32}{765}{AQ7}
\hand!-{KQ984}  {Q86}  {62}  {Q54}
\begin{bidding}-
& & 1D & X \\
p & ?\\
\end{bidding}
\expertquiz[]{%
2\Sp}


2\Sp; una scelta fatta con un salto rappresenta mani che stanno nella terra di mezzo:
senza la certezza di avere manche, ma “contente” del contratto che stanno
dichiarando. E’ un piccolo … segno di vita; mediamente con una dichiarazione a
salto promettete 8-10 punti e un colore di 5+ carte.

\section{L’intervento di 1\SA}
Promette forza e distribuzione equivalente all’Apertura di 1\SA. Visto che gli
avversari hanno già un’indicazione per l’attacco (c’è stata l’apertura in un colore)
l’intervento di 1\SA garantisce il fermo nel colore dell’Apertore. Si prosegue
esattamente come se avesse aperto di 1\SA (2\Cl Stayman, ecc ecc.).


\subsubsection{Esempi}


\newgame\setdefaults{bidfirst=N}
\dealer[\east]\vulner[\eastwest]
%\hand!-{Q952}{K32}{765}{AQ7}
\hand!-{AJ5} {KJ7} {AJT53} {K8}
\begin{bidding}-
& & 1H & ? \\
\end{bidding}
\expertquiz[]{%
1\SA}

\newgame\setdefaults{bidfirst=N}
\dealer[\east]\vulner[\eastwest]
%\hand!-{Q952}{K32}{765}{AQ7}
\hand!-{AK5} {98} {AK73} {QJ97}
\begin{bidding}-
& & 1H & ? \\
\end{bidding}
\expertquiz[]{%
\double}

\section{L’intervento a colore}
L’intervento a colore ha lo scopo di mettere in luce il tratto più significativo della
mano. Ha precisi limiti superiori: promette al massimo 16/17. La forza minima
promessa è 8 punti se a livello uno, e un po’ di più se a livello due.

Si interviene dichiarando un colore solo possedendo uno dei seguenti requisiti:
\begin{itemize}
\item Si possiede un ottimo colore, lungo e solido. Anche con soli 7/8 punti.
\item Si possiede un punteggio elevato, inadatto al Contro e a 1\SA. In tal caso è
possibile che il colore sia scarsamente onorato.
\end{itemize}

\begin{regola}{Massima per l'intervento}
Un intervento a colore è giustificato o da un ottimo colore o da tanti punti:
evitatelo se non possedete né un requisito né l’altro.
\end{regola}

\subsection{Un colore uno su uno}

\begin{center}

\begin{bidding}
    1D & 1S & \ldots & \ldots\\
\end{bidding}\qquad
\begin{bidding}
    1C & p & 1D & 1H\\
\end{bidding}

\end{center}
    \begin{tabular}{ll}
        Punti: & 8-16\\
        Carte:&
5 con almeno 1 Onore\\
              &
6 qualsiasi
    \end{tabular}

Se il punteggio è al minimo (7/8) è tutto concentrato nel colore.
Per Onore si intendono gli Onori Maggiori: Asso/Re/Dama.
Stessi requisiti in 2\textsuperscript{a} e in 4\textsuperscript{a} posizione.
\subsubsection{Esempi}

\begin{bidding}*
    1D & ?\\
\end{bidding}

\begin{tabularx}{\textwidth}{XXXX}
    \hand!{82} {AKJT42} {86} {983}&
    \hand!{K3} {Q7542} {985} {QJ6}&
    \hand!{K74} {AQJ63} {9} {KQ72}&
\hand!{AKQT3}{74}{K3}{QJ54}\\
    \qquad1\He & \quad Passo & \qquad1\He & \qquad1\Sp
\end{tabularx}

\subsection{Un colore due su uno}

\begin{center}
\begin{bidding}
    1D & 2C & \ldots & \ldots\\
\end{bidding}\qquad
\begin{bidding}
    1C & p & 1S & 2H\\
\end{bidding}
\end{center}

    \begin{tabular}{ll}
        Punti: & 10-16\\
        Carte:&
6 con almeno 1 Onore\\
              &
5 (solo se \He con almeno 1 Onore e 13+ punti)\
    \end{tabular}

L’intervento a colore 2 su 1 normalmente è almeno sesto, si fa eccezione per le
Cuori (nel caso di apertura 1 Picche e intervento di 2 Cuori) data la loro nobiltà.
I requisiti sono uguali sia in seconda che in quarta posizione.
\subsubsection{Esempi}

\begin{bidding}*
    1S & ? \\
\end{bidding}

\begin{tabularx}{\textwidth}{XXXX}
    \hand!{54} {98} {KQJ862} {A76}&
        \hand!{6} {J6} {AJ9853} {KQJ7}&
        \hand!{K8} {AJT63} {AQ76} {94}&
        \hand!{K98} {87} {KQ4} {AJ543}\\
       \qquad 2\Di & \qquad2\Di & \qquad2\He & \quad Passo
    \end{tabularx}

    \subsection{Come deve comportarsi il compagno}
Il compagno di chi ha fatto un intervento si baserà sul minimo garantito. Con 9+
punti non dovrà mai dire passo, perché la manche è possibile (16+9).
Gli appoggi possono essere dati anche con tre sole carte.

\subsubsection{Consigli}

Quando c’è competizione, nel 70\% dei casi il contratto finisce alla linea
dell’Apertore, per cui possiamo dire che ogni intervento risulterà avere influenza nel
controgioco. Se fate un intervento, e poi tocca a voi attaccare, potete anche
scegliere di attaccare in un altro colore, ma sappiate che il vostro compagno si farà
invece guidare dalla vostra informazione.

\begin{regola}{Intervento e attacco del compagno}
Quando state per dire un colore ricordatevi che il compagno
ci attaccherà. Se questo vi fa rizzare i capelli vuol dire
che l’intervento è sbagliato.
\end{regola}

\chapter{Dopo l'intervento avversario}

\section{Dichiarazioni obbligate, dichiarazioni libere}

Nel dialogo “a due” ogni dichiarazione fornita è la condizione indispensabile per
dare al compagno l’opportunità di dichiarare ancora: è un rimbalzo simile a un
palleggio su un campo da tennis, in cui due si tirano la palla e non c’è niente che
faccia “sponda”. Ma quando nella dichiarazione si intromette un avversario, crea
automaticamente una situazione in cui “l’ultima parola” spetta a chi è seduto alla
sua destra:

\begin{bidding}*
    1D & 1S & ?\\
\end{bidding}


Nord potrebbe avere 20 punti, ed è questo il motivo per cui Sud,
anche con 5, ha il compito di “tenere aperto”. Ma ora non serve
che lo faccia, l’intervento di Est darà modo comunque a Nord,
se ha davvero tanto, di rientrare in dichiarazione. Se Sud
fornisce una qualsiasi dichiarazione lo fa quindi \emph{liberamente}.

\begin{regola}{Dichiarazioni libere}
Una dichiarazione fornita liberamente,
in assenza di obblighi, mostra sempre una mano
che è contenta di dichiarare (una mano non minima, comprese anche mani particolarmente sbilanciate).
\end{regola}

\section{L’ intervento modifica le risposte}
Quando l’avversario secondo di mano interviene, le dichiarazioni del compagno
dell’Apertore subiscono delle modifiche.
\begin{enumerate}
\item non è più “necessario” parlare con 4-5 punti, quindi con mani nulle si passa.
\item i “senza” non sono più una risposta d’obbligo, ma proposta di contratto. E promettono il fermo nel colore avversario.
\item qualunque colore nuovo è almeno quinto.
\item si “aggiungono” Contro” (su un colore) e “Surcontro” (sul Contro).
\end{enumerate}

Gli appoggi rimangono uguali, come se l’intervento non ci fosse stato. Vediamo in
dettaglio.

\subsection{Su intervento avversario di Contro}
Con il Contro l’avversario propone di aggiudicarsi un contratto sulla sua linea; se
avete almeno 11, siete certi che il diritto a giocare sia vostro, e altrettanto certi che
la vostra linea è in grado di contrare (punitivamente) qualunque contratto alternativo
gli altri propongano. La dichiarazione che racconta tutto questo al compagno è una
sola: Surcontro.

\begin{regola}{Dichiarazione dopo intervento di Contro}
Dopo intervento avversario di contro vale che:
\begin{itemize}
\item Un colore nuovo è non forzante, e almeno quinto
\item Surcontro mostra 11+ e qualunque tipo di mano
\end{itemize}
\end{regola}

Il Surcontro è l’unica dichiarazione forte; la descrizione si rimanda al giro dopo.
Non esclude né promette fit. Le dichiarazioni alternative sono:
\begin{itemize}
\item Un colore nuovo: non è forzante (proposta di contratto finale), richiede un buon colore di almeno 5-6 buone carte.
    Perché 4 carte non bastano più? Perché si sta proponendo un colore in cui teoricamente l’avversario ha detto di
    voler giocare, quindi 5 carte sono una minima garanzia. I punti sono meno di 11.
\item Tutti gli appoggi: hanno lo
    stesso valore “come senza intervento”, ma in ogni caso meno di 11 punti.
\item 1\SA: 7--10 punti, contenti di giocarlo (fermi nei colori impliciti del contrante).
    \item Passo: tutte le mani non adatte a nessuna delle dichiarazioni scelte sopra.
\end{itemize}

\newpage
\subsubsection{Esempi}

\begin{bidding}*
    1H & X & ? \\
\end{bidding}

\hand*{AJxx} {Kx} {Qxx} {Qxxx}

Surcontro
\smallskip

\hand*{Kx}  {Jx}  {AQx}  {KQJTxx}

Surcontro (e non 2\Cl: essendoci il contro di mezzo, mostrereste una mano inferiore agli 11 e chiedereste al compagno di passare)
\smallskip

\hand*{xx}  {Jx}  {xxx}  {KQTxxx}

2\Cl: che significa “sarei contento di giocare 2\Cl!”
\smallskip

\hand*{AJx}  {xx}  {QTxx}  {JTxx}

1\SA: “non ho fit, e penso di mantenere il contratto di 1{\smaller SA}”


\subsection{Su intervento a Colore}

Un intervento a colore dell’avversario crea per il compagno dell’Apertore un vincolo
sulla lunghezza dei colori che vorrebbe descrivere: un colore nuovo infatti implica il
possesso di almeno 5 carte.

E per trovare i fit 4-4? L’uso del Contro con il significato di “vorrei giocare nei colori
rimanenti” appartiene a chi interviene sull’apertura ma anche, con identico
significato, a chi è seduto davanti all’Apertore:

\begin{regola}{Dopo intervento a colore}
Dopo intervento avversario a colore vale che:
\begin{itemize}
\item Un colore nuovo è forzante, almeno quinto
\item Contro cerca i fit 4-4 e chiede all’Apertore di dichiarare
\end{itemize}
\end{regola}

\begin{center}
\begin{bidding}
    1D & 1S & X & \ldots \\
\end{bidding}\qquad
\begin{bidding}
    1H & 1C & X & \ldots\\
\end{bidding}
\end{center}

In entrambi i casi Nord sta dicendo: “cerco fit nei colori che rimangono”.
Cuori e Fiori nel primo esempio, e Picche e Quadri nel secondo. I requisiti in forza
onori, visto che esiste già la garanzia dell’apertura del compagno, sono molto
inferiori a quelli dell’intervento (12+).

\begin{regola}{Il Contro del rispondente}
Il Contro promette 8+ punti (illimitato!)
e mostra “licita impedita”: chiede all’Apertore
di dichiarare altri colori, o i senza, se può.
\end{regola}

In sostanza dice “gioco nei pali rimasti. Tu riparla, e poi vediamo”. Non ha limiti
superiori: da 8 in su. Più la mano è forte, meno è vincolata al possesso di reale
lunghezza nei colori in cui si sta implicitamente cercando fit.

\subsubsection{Esempi}
\begin{bidding}*
    1D & 1S & ?\\
\end{bidding}

\hand*{xx} {AQxxx}  {Qx}  {KJxx}

2\He: tutto in regola, il colore è quinto e i punti consentono di andare a livello due.
\smallskip

\hand*{xx}  {AQxx}  {Qxx}  {KJxx}

Contro: i punti per dire un colore a livello 2 ci sono, ma è la lunghezza che manca.  Invece di dire un colore chiediamo
all’Apertore di mostrare altre quarte, se ne ha.
\smallskip

\hand*{xx}  {AQxx}  {QTxx}  {Qxx}

Contro: il fit nel minore lo abbiamo, ma prima di appoggiarlo proviamo a cercare contratti migliori. Se l’Apertore non
mostrerà le Cuori e se non potrà dire 1\SA, ci
rassegneremo a giocare a Quadri.

\hand*{xxx}{Kxx}{AQxx}{KJx}

Contro: la mano non è perfetta dal punto di vista distribuzionale (le Fiori ci sono, le
Cuori sono soltanto tre) ma i punti abbondanti compensano questa carenza.

Le alternative al Contro sono:
\begin{itemize}
\item Un colore nuovo. \`E forzante, almeno quinto; promette almeno 7/8 punti se a livello
1, almeno 11 se a livello 2.
\end{itemize}

\newpage

\subsubsection{Esempi}
\begin{bidding}*
    1D & 1H & ?\\
\end{bidding}

\begin{tabularx}{\textwidth}{>{\raggedright\arraybackslash}XXXX}
    \hand!{AQ73} {87} {Q4} {AQ764}&
\hand!{KQ763} {87} {Q43} {J76}&
\hand!{AKJ63} {87} {A43} {Q76}&
\hand!{K763} {87} {43} {AQ764}\\
\qquad2\Cl & \qquad 1\Sp & \qquad1\Sp & \qquad Dbl\sidefootnotemark
\end{tabularx}
\sidefootnotetext{Questa è la dicitura contratta
dell'inglese \emph{Double} (raddoppio, infatti i valori delle prese, anche quelle in meno, raddoppiano), l'equivalente dell'italiano Contro.}

\begin{itemize}
\item Gli appoggi, che restano invariati.
\item I Senza, che promettono il fermo nel seme avversario (1\SA: 7-10, 2\SA: 11-12, 3\SA: 13-14).
\end{itemize}

Ricapitolando, il Contro supplisce a tutte le situazioni in cui non si può fare una
dichiarazione “naturale”, o perché pur essendoci una lunga mancano i punti per dirla
al livello necessario, o perché pur essendoci i punti manca lunghezza nel colore.

Sul Contro l’Apertore dichiara altri colori, se ne ha, o si arrangia dicendo
comunque qualcosa. Poiché obbedisce a un ordine, eventuali dichiarazioni a colore
nuovo che superino “2 nel primo palo” NON mostrano mano di rovescio:

\begin{bidding}*
    1D & 1S & X & p\\
    2H & \ldots & \ldots & \ldots\\
\end{bidding}

Nord, che ha dichiarato per obbligo, può avere:

\hand*{xx} {AJxx} {KQxx} {Kxx}

\section{L’ intervento del quarto: cosa fa l’Apertore}

Anche l’Apertore, dopo intervento alla sua destra, si può trovare in una situazione di
“parlata libera”:

\begin{bidding}*
    1D & p & 1H & 1S\\
    ? \\
\end{bidding}

Sud (1\He forzante) ha chiesto di poter
ridichiarare, ma l’intervento di Est gli
consente di farlo anche se Sud scegliesse
di dire Passo. Quindi, se Sud dichiara, lo fa
“liberamente”.

\begin{regola}{Dichiarazioni dell'Apertore su intervento del quarto di mano}

Le regole sono molto semplici:
\begin{itemize}
\item Con tutte le mani “normali”, bilanciate o sbilanciate, l’Apertore Passa. A meno che abbia fit quarto: in tal caso appoggia il compagno, proporzionalmente alla propria forza.
\item Con tutte le sbilanciate di Diritto che presentano o un’ottima monocolore o una bicolore di almeno 10 carte, ridichiara. A condizione, ovviamente, di restare entro il livello di guardia.
\item Con le sbilanciate di Rovescio dichiara normalmente, ignorando l’intervento.
\item Con la bilanciata 18-20, se ha il fermo, dichiara i Senza al minimo livello.
\end{itemize}
\end{regola}

\subsubsection{Esempi}

\begin{bidding}*
    1D & p & 1H & 1S\\
    ?\\
\end{bidding}

\hand*{Jx} {Kx} {AQxxx} {Qxxx}

Passo; una bicolore al minimo dei requisiti (di punti e di distribuzione) è una mano
“normale”!
\smallskip

\hand*{x} {xx} {AKJxx} {KQJxx}

2\Cl: il punteggio è di Diritto, ma ci sono ottimi colori da dire velocemente, prima che
le Picche vengano rialzate da Ovest.
\smallskip

\hand*{Kx} {xx} {Kxxxxx} {AQx}

Passo; non è così pressante la voglia di informare il compagno che le Quadri sono
tante, vista la loro povertà. Ben diverso con:
\smallskip

\hand*{xx} {xx} {AKJTxx} {KQx}


2\Di: un colore ridichiarato spontaneamente è sempre un “signor colore”.
\smallskip

\hand*{xx} {KJxx} {KQxx} {Kxx}

2\He: il fit è sacro, e consente di sopravvivere in un contratto di 8 prese anche se la
coppia avesse solo 18 in linea.
\smallskip

\hand*{AQx} {Jxx} {Kxxx} {Kxx}

Passo: senza intervento avremmo dichiarato 1\SA, ma questa è una mano normale,
anche se ha il fermo a Picche.
\smallskip

\hand*{AQx} {Kxx} {AKxx} {Kxx}

1\SA: non c’è bisogno del salto a 2\SA per chiarire la bilanciata 18-20, perché non
possono esserci equivoci: con 15-17 avremmo aperto 1\SA, e poiché con 12-14
adesso diremmo Passo, questa replica non può mostrare altro che questa mano.

Anche se l’annuncio di un nuovo colore può restare a livello uno, rimane il fatto che
l’Apertore dichiara libero solo con mani che abbiano quanto meno una buona
distribuzione. Quindi:

\begin{bidding}*
    1C & p & 1D & 1H \\
    1S\\
\end{bidding}

Che carte dobbiamo aspettarci da Nord?
Qualcosa tipo questo:

\hand*{AQxx} {x} {Qxx} {KQxxx}

\noindent
la dichiarazione “libera” di 1\Sp rende reali e lunghe le Fiori, visto che la mano deve
essere sbilanciata (e Fiori è l’unica lunga possibile!) . Non certo queste:
\smallskip

\hand*{AQxx} {Kxx} {Qxx} {Qxx}

\noindent
con cui si dice Passo.

\chapter*{Etica e Regole}
\addcontentsline{toc}{chapter}{Etica e Regole}
\markboth{Etica e Regole}{Etica e Regole}

\begin{enumerate}
\item Una carta giocata non si cambia più, né ci si prova.
\item Non si può chiedere di rivedere le carte giocate, a meno che non si abbia la propria carta ancora scoperta sul tavolo. Nessuno può giocare su una presa successiva finché uno dei giocatori tiene ancora la sua carta scoperta; se l’avversario gioca troppo veloce, questo è il modo con cui potete imporre il ritmo che vi consente di non andare in confusione: aspettate qualche secondo prima di girare la vostra.
\item Il board deve restare in centro tavolo per tutta la smazzata; questo evita imbussolamenti errati.
\item Quando scende il Morto è buona norma dire “grazie” al compagno. Siate costanti in questa pratica, anche perché il
    non farlo può suggerire all'avversario che siete in difficoltà.
\item I difensori non devono mai preparare la carta prima che sia il loro turno di gioco.
\item Il silenzio, assoluto, durante tutta la smazzata, è la più ferrea regola del bridge.  Anche se vi accorgete di avere sbagliato, non ditelo: vi scuserete dopo. Non avete il diritto di mettere in guardia il vostro compagno perché possa parare il colpo. Nessun tipo di commento deve essere fatto, né dovete lasciar trapelare emozioni.
\item Nessuno può toccare le carte di un altro giocatore (tranne che quelle del Morto, con l’intento di sistemarle meglio). Non è vietato che il Giocante muova personalmente le carte del Morto, ma non è conveniente; facendolo è facile sporgersi sul tavolo, e dare agli avversari una bella panoramica delle proprie carte.
\item I messaggi che due compagni si scambiano sono dati dalle dichiarazioni e dalle carte giocate. Tutto il resto: commenti, sbuffi, smorfie, sorrisi sono informazioni illecite, e nel Bridge sono sanzionati.
\item Non è corretto giocherellare con i cartellini del bidding box. Che la vostra mano parta quando il cervello ha già deciso cosa dichiarare.
\item Il Codice è stato inventato per porre rimedio a situazioni che hanno involontariamente alterato il normale svolgimento del gioco, e il Direttore ne è l'esperto e l'unico autorizzato ad applicare le eventuali sanzioni. Chiamatelo senza timore, è lì apposta per rimediare. E non è il caso di offendersi se qualcuno lo chiama per una vostra vera o presunta irregolarità.
\item Il Morto \`E MORTO, non si muove dal suo posto, non suggerisce e non tocca le sue carte (neanche se si tratta di
    un singolo) fintantoché il Giocante non le chiama. Tiene il conto delle prese vinte o perse e fa il possibile perché
    il vivo possa restare concentrato su quello che fa.
\item Il tempo è di tutti: non sprecatelo tergiversando in situazioni in cui pensare non serve, o non serve più. Più in
    fretta giocate, più mani giocate e più vi divertite. Se potete fare claim da Giocante, ovvero affermare con sicurezza il numero
    di prese che fate e cedete agli avversari, fatelo: mostrate le carte e illustrate la vostra linea di gioco. Se gli
    avversari sono d'accordo risparmierete tempo e fatica. Ma attenzione, dovete essere sicuri del numero di prese che
    farete, altrimenti l'avversario potrà controgiocare avendo visto le vostre carte. Se da difensori non siete sicuri
    di un claim avversario, chiedete gentilmente di continuare a giocare la mano.
\end{enumerate}
\end{document}
