\documentclass[../corsofiori.tex]{subfiles}
\setcounter{regolactr}{25}
\setcounter{chapter}{7}
\begin{document}

\section{Decidere cosa dichiarare}
\setdefaults{colors=4A}

Quando è il nostro turno di dichiarare, seguiamo la seguente procedura per scegliere la nostra dichiarazione.

\paragraph*{Se nessuno ha aperto}

\begin{enumerate}
    \item Se abbiamo meno di 12 punti, passiamo
    \item Se abbiamo tra i 12 e i 20 punti, apriamo a livello 1:
        \begin{enumerate}
            \item Se abbiamo una mano sbilanciata:
                \begin{enumerate}
                    \item Se abbiamo 5\Sp, apriamo 1\Sp;
                    \item Se abbiamo 5\He, apriamo 1\He;
                    \item Se abbiamo 4\Di, apriamo 1\Di;
                    \item Altrimenti, apriamo 1\Cl
                \end{enumerate}
            \item Se abbiamo una mano bilanciata:
                \begin{enumerate}
                \item Se abbiamo 15--17 punti, apriamo 1\SA;
                \item Se abbiamo 12--14 o 18--20, apriamo come se avessimo una sbilanciata, ma:
                    \begin{enumerate}
                   \item Al giro successivo, dichiareremo 1\SA se abbiamo 12--14;
                \item Dichiareremo 2\SA se abbiamo 18--20
                    \end{enumerate}
                \end{enumerate}
        \end{enumerate}
        \item Se abbiamo più di 20 punti:
            \begin{enumerate}
                \item Se abbiamo una mano bilanciata:
                    \begin{enumerate}
                        \item Apriamo 2\SA se abbiamo 21--22 punti;
                        \item Apriamo 2\Cl se abbiamo 23+ punti, per poi dichiarare 2\SA;
                    \end{enumerate}
                \item Se abbiamo una mano sbilanciata:
                    \begin{enumerate}
                        \item Apriamo nello stesso colore con cui avremmo aperto se avessimo avuto 12--20 punti.
                    \end{enumerate}

            \end{enumerate}
\end{enumerate}

\paragraph*{Se il nostro compagno ha aperto}
\begin{enumerate}
    \item Contiamo i punti che possediamo sulla linea, ovvero in coppia.
        \item In base al giustificativo che siamo in grado di raggiungere, scegliamo il contratto da dichiarare secondo
            la seguente tabella.

\begin{tabular}{llc}
       \toprule
    Giustificativo & Denominazione contratto & Contratto\\
    \midrule
    con 20 - 24 punti coppia&contratti parziali& tutti, tranne i seguenti\\
con 25 - 31&contratti di Manche            &3NT, 4\Sp, 4\He, 5\Di, 5\Cl \\
con 32 - 36&contratti di piccolo Slam      &6NT, 6\Sp, 6\He, 6\Di, 6\Cl \\
con 37 o più&contratti di grande Slam      &7NT, 7\Sp, 7\He, 7\Di, 7\Cl\\
\bottomrule


\end{tabular}

\item Se possediamo un fit in un colore maggiore (\Sp o \He), dichiariamo il contratto nel maggiore.
    \item Se non lo possediamo, dichiariamo il contratto a \SA (per il momento).
\end{enumerate}

\paragraph*{Se l'avversario ha aperto}

\begin{enumerate}
    \item Se possediamo meno di 7--8 punti, passiamo;
    \item Se possediamo più di 8--9 punti:
        \begin{enumerate}
            \item Se possediamo un bel colore almeno quinto, lo dichiariamo;
            \item Altrimenti, passiamo
        \end{enumerate}
\end{enumerate}

\chapter{Le aperture di 1\SA e 2\SA}
Le aperture di 1\SA (15--17) e 2\SA (21--22) sono aperture speciali riservate alle mani bilanciate (\textbf{4333},
\textbf{4432}, \textbf{5332}). Di conseguenza, promettono almeno due carte in ogni colore.

\section{L'apertura di 1\SA}

Si apre di 1\SA con 15,16 e 17 punti. Se una mano è bilanciata in questa fascia di punteggio, allora non va aperta di
1 a colore.

Essendo una apertura molto descrittiva (si sanno molte cose della mano dell'Apertore), di conseguenza, è il
Rispondente\sidefootnote{Si definisce Rispondente il compagno dell'Apertore}
a decidere il contratto. Egli conterà i propri punti e, sommandoli ai 15--17 dell'Apertore, avrà un'idea precisa del
livello del contratto. Se ritiene che la manche sia irraggiungibile, può dichiarare Passo per giocare 1\SA.

Vediamo alcune situazioni, partendo dal presupposto che l'Apertore abbia aperto di 1\SA e noi siamo il Rispondente.

\begin{wraptable}[5]{l}[-1cm]{3cm}
\hand!{AQ8743} {K5} {J42} {76}
\end{wraptable}

L'Apertore ha almeno 15 punti e al massimo 17, inoltre, ha almeno 2\Sp. Sulla linea abbiamo, quindi,
25--27 punti e almeno 8\Sp, quindi il \textsc{fit}. I contratti in atout sono la nostra prima scelta, quindi
dichiareremo 4\Sp.

\begin{wraptable}[4]{l}[-1cm]{3cm}
\hand!{J3} {K52} {AQJ52} {J76}
\end{wraptable}

Dichiariamo 3\SA: i punti bastano
(27-29) ed un fit maggiore è
improbabile. Le Quadri daranno molte
prese, ma il contratto deve essere a
\SA, non a Quadri (economicamente non conviene).

\begin{regola}{Dichiarazioni di manche su 1\SA}
    Sull'apertura di 1\SA, le dichiarazioni di contratti di manche da parte del Rispondente sono conclusive.
\end{regola}

\begin{wraptable}[5]{l}[-1cm]{3cm}
\hand!{876} {JT9652} {42} {76}
\end{wraptable}

La mano del Rispondente non vale nulla giocando a \SA! Le \cu sono, infatti, irraggiungibili anche se affrancate. Di
conseguenza, cercheremo di sfruttare le \cu come atout: dichiareremo, quindi 2\He. La scelta di giocare a colore
a livello 2 si può fare anche quando si hanno sole 5 carte in quel seme, ma mai con solo 4 carte. Il punteggio richiesto
parte da zero:

\hand*{x}{xxxxx}{xxxx}{xxx}

Su 1\SA, il contratto di 2\He probabilmente non verrà mantenuto, ma 1\SA sarebbe peggio.


\begin{regola}{Dichiarazioni che corrispondono a parziale minimo}
    Sull'apertura di 1\SA, le dichiarazioni di contratti parziali minimi, quindi a livello 2 (2\Di, 2\He, 2\Sp), da parte del Rispondente sono conclusive.
\end{regola}

\begin{wraptable}[2]{l}[-1cm]{3cm}
\hand!{876} {AJ8652} {Q2} {76}
\end{wraptable}

Il Rispondente dichiarerà 3\BH: si gioca a \cu, ma vuole che l'Apertore rialzi solo con 17 punti.


\vspace{1cm}

\begin{wraptable}[3]{l}[-1cm]{3cm}
\hand!{Q76} {K2} {QJ72} {762}
\end{wraptable}


Dichiarate 2\SA: significa che si possiede più del Passo, ma meno del 3\SA. Chiede al compagno di rialzare con il
massimo o passare con il minimo.

\vspace{1cm}

\begin{regola}{Dichiarazioni di parziali non minimi su 1\SA}
    Sull'apertura di 1\SA, le dichiarazioni di parziali non minime (2\SA, 3\Cl,3\Di,3\He,3\Sp), quindi inutilmente alte,
    mostrano certezza di fit (quindi almeno 6 carte nel colore dichiarato) e incertezza
    sul livello a cui giocare, e chiedono all'Apertore di dichiarare 4 con il massimo del punteggio.
\end{regola}

Se il Rispondente non è in grado di decidere il contratto perché non ha informazioni a sufficienza, può usare una
dichiarazione speciale per chiedere quali colori nobili quarti possieda: la dichiarazione di 2\Cl \emph{Stayman} (dal
nome dell'inventore).

Questa dichiarazione interrogativa convenzionale chiede al Rispondente la distribuzione. Si può usare quando si
possiedono almeno 8 punti e 4/5 carte in un colore nobile. Con 6 carte non è necessaria, dato che abbiamo sicuramente il
fit, e neanche con meno di 4, in quanto è improbabile trovare un fit in un maggiore.

L'Apertore risponde nel seguente modo:
\begin{itemize}
    \item 2\Di se non possiede 4/5 carte né a \cu né a \pic.
    \item 2\He se possiede 4/5\He e non 4\He.
    \item 2\Sp se possiede 4/5\Sp e non 4\Sp.
    \item 2\SA se possiede 4\Sp4\He e il minimo.
    \item 3\Cl se possiede 4\Sp4\He e il massimo.
\end{itemize}

A questo punto ci sono 3 alternative:
\begin{enumerate}
    \item \`E stato individuato un fit, e allora il Rispondente dichiara 3 nel maggiore per chiedere all'Apertore di
        dichiarare 4 con il massimo, o dichiara direttamente manche.
    \item Sicuramente non c'è il fit, allora il Rispondente dichiara i \SA proporzionalmente alla sua forza.
    \item Fit 4/4 non trovato, ma possibile fit 5/3 perché il Rispondente ha 5 carte: in questo caso il Rispondente
        dichiara la sua quinta. Se l'Apertore ha 3 carte, rialza il colore, altrimenti ripiega a \SA.
\end{enumerate}

Esempi:

\newgame
\boardnr{1}
\westhand{J7} {KQ96} {AQ2} {AJ72}
\easthand{Q942} {JT42} {K5} {KQ3}
\showEW\qquad
\begin{biddingpair}(\explainit{Che quarte maggiori hai?\\}%
    \explainit{Ho 4\He\\} \explainit{Anche io. Siamo arrivati})
    1N & 2C\markit\\
    2H\markit & 4H\markit\\
\end{biddingpair}
\bigskip

\newgame
\boardnr{2}
\westhand{AQ75} {76} {KQ4} {AJ52}
\easthand{64} {KQ42} {J965} {KQ3}
\showEW\qquad
\begin{biddingpair}(\explainit{Che quarte maggiori hai?\\}%
    \explainit{Ho 4\Sp\\} \explainit{Io 4\He. Giochiamo 3\SA})
    1N & 2C\markit\\
    2S\markit & 3N\markit\\
\end{biddingpair}

\bigskip

\newgame
\boardnr{3}
\westhand{K5} {KQ76} {AJ4} {A852}
\easthand{QJ976} {A2} {Q965} {K3}
\showEW\qquad
\begin{biddingpair}(\explainit{Che quarte maggiori hai?\\}%
    \explainit{Ho 4\He\\} \explainit{Io 5\Sp\\}
    \explainit{Non ho 3\Sp\\})
    1N & 2C\markit\\
    2H\markit & 2S\markit\\
    2N\markit & 3N\\
\end{biddingpair}

\section{L'apertura di 2\SA}
Si apre di 2\SA con 21--22 punti e le medesime distribuzioni con cui apriamo 1\SA. Se una mano ha i requisiti per aprire
di 2\SA, questa apertura prevale sulle aperture di 2 a colore.

\paragraph*{Risposte}

L'unico modo per fermarsi prima di manche è Passo. In alternativa:

\begin{itemize}
    \item Tutte le dichiarazioni di manche o slam sono conclusive;
    \item Le risposte di 3 a colore (3\Di, 3\He, 3\Sp) mostrano 5+ carte in quel colore e forza almeno di manche.
        L'Apertore non dovrà mai passare. Se ha fit rialza a manche, altrimenti riporta a 3\SA.
    \item Se il Rispondente cerca un fit 4/4, usa l'interrogativa 3\Cl. Le risposte, scalate di un livello, sono le
        stesse dell'apertura 1\SA (più o meno):
        \begin{itemize}
            \item 3\Di: né 4\He, né 4\Sp.
            \item 3\He: 4\He.
            \item 3\Sp: 4\Sp.
            \item 3\SA: sia 4\Sp che 4\He.
        \end{itemize}
\end{itemize}



\end{document}
