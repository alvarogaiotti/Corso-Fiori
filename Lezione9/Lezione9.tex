\documentclass[../corsofiori.tex]{subfiles}
\setcounter{regolactr}{28}
\setcounter{chapter}{8}
%\externaldocument[VIII-]{../lezione8/Lezione8}
\begin{document}
\setdefaults{colors=4A}
\chapter{Rispondere all'apertura di 1 a colore}

Apertore e Rispondente, nella fase dichiarativa, si scambiano informazioni con un duplice scopo: appurare le lunghezze
dei colori alla ricerca di un fit, e appurare la forza combinata delle due mani per decidere il livello a cui giocare.
Poiché indicativamente i 24/25 punti coppia danno ragionevoli probabilità di realizzare un contratto di manche, nessuno
dei due dovrà abbandonare (ovvero Passare) finché non avrà potuto escludere con certezza che la coppia possieda tale
requisito.

\section{Risposte limitative, invitanti e forzanti}
\`E fondamentale che il Rispondente sappia, fin dalla sua prima occasione dichiarativa, quale sia il valore dinamico
della risposta che fornisce; in pratica sapere se la sua licita può indurre a un arresto del dialogo (Passo del
compagno) o se obbligatoriamente ne provocherà la continuazione. Alcune risposte infatti mostrano precisi limiti,
a fronte dei quali è possibile che l’Apertore decida di passare (risposte limitative); altre, pur mostrando un limite,
esprimono il desiderio di giocare manche a condizione che l’Apertore abbia qualcosa in più del minimo (risposte
invitanti).

Il terzo tipo di risposta possibile, che è costituito da qualsiasi cambio di colore, richiede all'Apertore di permettere
al Rispondente di dichiarare nuovamente, quindi obbliga tassativamente
l’Apertore (se gli avversari tacciono) a fornire una seconda descrizione (risposte forzanti).

Una risposta di un nuovo colore assicura un punteggio minimo, ma nulla dice sul massimo: per questo motivo l’Apertore
deve fare in modo che il compagno abbia ancora la possibilità di ridichiarare.

\subsection{Il Passo (0--4 punti)}

\begin{regola}{Passo sull'apertura}
    Quando il compagno apre di 1 a colore, non è permesso passare se si possiedono 5+ punti, in quanto il compagno
    potrebbe avere 20 punti e potrebbe esserci la manche.
\end{regola}

Se abbiamo meno di 5 punti, dato che l’Apertore ha tra i 12 e i 20 punti, sappiamo che non è possibile fare manche. Di
conseguenza, è ragionevole passare al primo parziale disponibile. Quando il Rispondente ha, invece, più di 5 punti la
manche non è esclusa e, pertanto, egli dovrà \emph{sempre} dichiarare.

\subsection{Gli appoggi (5--11 punti)}

Tutte le risposte di appoggio (rialzo di un parziale nel colore di apertura), mostrano un limite superiore e inferiore.
Questo mette l'Apertore nella condizione di avere un'idea precisa della forza della coppia e, di conseguenza, di
prendere decisioni, compresa quella di passare. Gli appoggi si dicono, in gergo bridgistico, \emph{non forzanti}, in
quanto l'Apertore non è obbligato (forzato) a dichiarare.

\subsubsection{Gli appoggi a livello 2}
Questi appoggi sono \emph{limitativi}, mostrano un numero di carte minimo sufficiente a garantire il fit (4 se
l'apertura è di 1\Di, 3 se è di 1\He/\Sp) e mostrano 5--10 punti. Il messaggio è che se l'Apertore ha una mano di forza
minima (12--13), la manche è irraggiungibile.

\begin{attenzione}{Eccezione}
    Sull'apertura di 1\Cl, la risposta di 2\Cl non è considerata un appoggio (in quanto l’Apertore potrebbe avere solo
    2\Cl, e quindi il colore di \fio non è \emph{reale}), bensì un nuovo colore, che, come vedremo in seguito, in quanto
    dichiarato a livello 2 mostra 12+ punti.
\end{attenzione}

\subsubsection{Gli appoggi a livello 3}
Questi appoggi sono \emph{invitanti}: mostrano fit e circa 11 punti. Ricordate la \autoref{VIII-reg:nonmin}? La
dichiarazione di un parziale non minimo mostra incertezza sul livello a cui giocare: manche o parziale. La manche sarà
possibile, quindi, se l’Apertore possiede almeno 14 punti.

In questo caso, la dichiarazione di 3\Cl è da considerarsi un appoggio, quindi con:

\hand*{3}{KJ5}{Q84}{AJ9854}

su 1\Cl, il Rispondente dichiarerà 3\Cl.

\subsubsection{Gli appoggi nel nobile a livello di manche}
Le dichiarazioni di 4\He su 1\He e 4\Sp su 1\Sp mostrano 4 carte ed una distribuzione che si presta a fare prese di
taglio. L'idea che trasmette all'Apertore è che si possiede una mano con cui sia possibile fare 10 prese anche a fronte
di un'apertura minima. Quando si appoggia a livello 4 si descrive una mano di forza limitata (mai più di 11 punti) che
si è notevolmente rivalutata nel momento il cui è stato trovato il fit e, quinti, vale di più per la grande capacità di
taglio.

Per quantificare questo incremento di forza, si usa la \textsc{Legge di Rivalutazione}:

\begin{regola}{Legge di Rivalutazione}
    Quando il Rispondente ha una mano sbilanciata e fit di almeno 4 carte, aggiunge ai suoi punti un valore dato dalla
    differenza tra il numero di carte di atout e il numero di carte nel colore più corto.
\end{regola}

Dopo aver applicato questo correttivo, appoggia in base al totale ottenuto:

\smallskip
\hand*{8}{KJ72}{KQ74}{7642}

        Su apertura 1\He: 9 punti onore, più 3 per la Legge di Rivalutazione: 12 $\Rightarrow$ 4\He.

        \smallskip
\hand*{A9852}{-}{T942}{KJ43}

Su apertura 1\Sp: 8 punti onore, più 5 per la Legge di Rivalutazione: 13 $\Rightarrow$ 4\Sp.

\subsubsection{Passaggio per un altro colore prima dell'appoggio a livello 4}
    Se avete fit ed una forza onori di 12+, non appoggiate immediatamente: dichiarate prima un altro colore e poi
    appoggiate il colore di apertura a livello 4, in modo da trasmettere il messaggio che possedete forza onori genuina
    e non rivalutata.

    Non vi preoccupate del fatto che il compagno potrebbe passare: come vedremo tra poco, la dichiarazione di un nuovo
    colore è \emph{forzante}.
    \smallskip

            \hand*!{KJ2}{Q2}{842}{AQJ43}

    \smallskip
        Su apertura 1\Sp, dichiarate 2\Cl; \emph{dopo} direte 4\Sp.


\subsection{I nuovi colori a livello 1 (5+ punti)}

Quando dichiara un nuovo colore a livello 1, il Rispondente promette almeno 5 punti ed un massimo illimitato. Chiede che
l'Apertore gli ritorni la parola: di conseguenza, egli non può passare se il suo passo potrebbe, anche solo
potenzialmente, portare al termine della dichiarazione.
I nuovi colori sono dichiarazioni \emph{forzanti}.

\smallskip
\noindent
\begin{bidding}
    1C & p & 1H & p\\
    p\\
\end{bidding}\quad
Il Passo è un errore gravissimo. Ma\ldots

\smallskip

\begin{wraptable}[4]{l}{4.445cm}
\begin{bidding}
    1C & p & 1H & 1S\\
    p\\
\end{bidding}
\end{wraptable}

\mbox{}\newline
Questa volta Sud ha detto 1\Sp, quindi Ovest non è obbligato a dichiarare, in quanto la parola tornerà ad Est
sicuramente.

\smallskip
Le risposte di \emph{uno su uno} (cambio di colore a livello 1) promettono 4+ carte nel seme
dichiarato\sidefootnote{Questo vale anche per le dichiarazioni di 1\He e 1\Sp, che in apertura mostrano 5+ carte e non
solo 4.} e chiedono
all'Apertore di descriversi ulteriormente, con precedenza assoluta per il fit.

Quando in dubbio sulla risposta da effettuare, in quanto possedete due colori di almeno 4 carte, seguite la seguente
regola:

\begin{regola}{Ordine di dichiarazione dei colori in risposta}
    Con due colori entrambi lunghi (5+ carte) dichiarate prima quello di rango più alto. Con un colore di 5+ carte e uno
    di 4, iniziate con il più lungo. Con due colori di sole 4 carte, iniziate con quello di rango inferiore.
\end{regola}

Quando il Rispondente ha fit nel minore dell'Apertore ma possiede anche una quarta nobile è importante che dichiari
prima il nobile, in quanto la manche in un seme maggiore è da preferire. Se il fit non viene trovato, ripiegherà allora
sul fit nel minore.

\hand{K986}{62}{KJ73}{862}$\Rightarrow$ 1\Sp; \emph{dopo} 2\Di.


\subsection{Nuovi colori a livello 2 (12+ punti)}

Quando il Rispondente dichiara un nuovo colore a livello 2, egli promette almeno 12 punti. Le risposte 2 su 1 sono,
quindi \emph{forzanti fino a manche}: in linea ci sono almeno 24 punti e non si può passare prima della dichiarazione
di un contratto almeno di manche.

Lo scopo delle dichiarazioni 2 su 1 è chiedere all'Apertore una descrizione generica. Il numero di carte richiesto per
effettuare una dichiarazione di 2/1\sidefootnote{2 su 1.} rispecchi gli stessi vincoli dell'apertura di 1: 5+ carte per 2\He/\Sp, 4+ per 2\Di,
2+ per 2\Cl.

Ricordatevi che con 12+ punti non è necessario dichiarare a livello 2 se la dichiarazione a livello 1 è possibile. Su
apertura di 1\Di, con:

\smallskip
\hand*{A4}{KQ952}{K97}{KQ3}

\noindent
rispondete 1\He, in quanto la dichiarazione di 2\He sarebbe un salto (salterebbe il gradino di 1\He) e, quindi, dotata
di significato differente (che vedremo più avanti). Quindi, non abbiate fretta a dichiarare a livello 2 se possedete
dichiarazioni più economiche nello stesso seme.

\subsection{La risposta di 1\SA (5--10 punti)}

Può succedere che il Rispondente si trovi nella condizione di non poter passare, eppure di non possedere un colore da
dichiarare a livello 1 e non possedere la forza per dichiarare a livello 2. A volte si dovrà rinunciare a mostrare un
colore. Su apertura 1\He, con:

\hand*{K732}{73}{Q7}{Q9852}

\noindent
dichiarerà 1\Sp nonostante le fiori siano più lunghe, dato che 2\Cl garantirebbe 12 o più punti.

Altre volte il Rispondente sarà nella condizione di non poter dichiarare nessun colore:

\hand*{32}{4}{QJ973}{KQ852}

\noindent
ben due quinte, ma sull'apertura di 1\He o 1\Sp si può dichiarare nessuno dei due colori, a causa della forza
insufficiente per una risposta a livello 2. In questi casi si usa la risposta di 1\SA per indicare il possesso della
forza necessaria a rispondere, la mancanza di colori di 4 carte dichiarabili a livello 1 e l'insufficienza di punti per
una risposta almeno invitante.

\begin{regola}{Risposta di 1\SA}
    La risposta di 1\SA, \emph{non forzante}, mostra 5--10 punti e nega 4+ carte in tutti i colori che avrebbero
    potuto essere dichiarati a livello 1.
\end{regola}

Trattandosi di una risposta di necessità, non promette una mano bilanciata:

\smallskip
\setdefaults{bidfirst=S}
\begin{biddingpair}
    1S & 1N\\
\end{biddingpair} \quad Nord ha 5--10 punti e qualsiasi distribuzione (non 3\Sp).

\smallskip
\begin{biddingpair}
    1H & 1N\\
\end{biddingpair} \quad Nord non ha né 4\Sp né 3\He, qualsiasi distribuzione, 5--10 punti.\\

\smallskip
\begin{biddingpair}
    1D & 1N\\
\end{biddingpair} \quad Nord non ha 4\Sp,4\He o 4\Di. Quindi ha almeno 4\Cl e 5--10 punti.\\


\smallskip
\begin{biddingpair}
    1C & 1N\\
\end{biddingpair} \quad Nord ha le stesse mani con cui dichiara 1\SA su 1\Di.

\subsection{Le risposte 2\SA (11 punti) e 3\SA (12--15 punti)}

La risposta di 2\SA è \emph{invitante} e ha le stesse caratteristiche di distribuzione della risposta di 1\SA, ma
mostra una mano che può fare manche con un apertura non minima (14 punti). Esclude possibilità di fit nei maggiori
e chiede all'Apertore di rialzare a 3\SA con un apertura buona.

La risposta di 3\SA mostra certezza di raggiungere i 25 punti in linea e mostra una mano bilanciata. Esclude fit nei
colori maggiori, ma è possibile il possesso di un fit in un minore se la mano è bilanciata. Ad esempio, su 1\Di, con:

\smallskip
\hand*{AQ6}{KJ4}{JT98}{QJ8}

\noindent si dichiara 3\SA: la manche è sicura e, in più, la mano è estremamente bilanciata.


\end{document}
