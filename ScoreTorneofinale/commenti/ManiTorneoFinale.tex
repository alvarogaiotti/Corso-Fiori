\PassOptionsToPackage{dvipsnames,x11names,svgnames}{xcolor}
\PassOptionsToPackage{colors=4A}{onedown}
\PassOptionsToPackage{usenames,dvipsnames,pdftex}{color}
\documentclass[a4paper,italian,12pt]{article}
\usepackage[margin=1.5cm]{geometry}
\usepackage[T1]{fontenc}
\usepackage[utf8]{inputenc}
\usepackage{newcent}
\usepackage{helvet}
\usepackage{graphicx}
%\usepackage{fancyhdr}
%\pagestyle{fancy}
\usepackage{wrapfig}

%%%%%%%%%%%%%%%%%%%%%%%%%%%%%%%%%%%%%%%%%%%%%%%

\newtheorem{theorem}{Principio }
\usepackage[colors=4A]{onedown}
\usepackage[italian]{babel}
\usepackage{enumitem}
\usepackage[most]{tcolorbox}
\usepackage{multicol}
\usepackage{xr}
%%% Colors
\definecolor{linkcolor}{HTML}{E23812}
\definecolor{azzurro}{HTML}{1ECBE1}
\definecolor{titlepagecolor}{HTML}{FF0800}
\definecolor{MidnightBlue}{HTML}{191970}
\definecolor{arancionediam}{HTML}{FFA500}
\definecolor{OLIVEGREEN}{rgb}{0.33, 0.42, 0.18}
\definecolor{MIDNIGHTBLUE}{HTML}{191970}
\definecolor{REDORANGE}{HTML}{FF5349}
\definecolor{RED2}{HTML}{EE0000}
%\definecolor{titlepagecolor}{RGB}{255,40,0}
%{cmyk}{1,.60,0,.40}

\usepackage[pdftex, pdfborder={0 0 0}, colorlinks=true, linkcolor=linkcolor]{hyperref}
\frenchspacing

\usepackage{txfonts} % For \varheartsuit and \vardiamondsuit
%\usepackage[usenames,dvipsnames,monochrome]{color} % dvipsnames necessary to made PDFLaTeX work.

\usepackage{listliketab}
\usepackage{latexsym} % \Box
\usepackage{pbox} % \Box
\usepackage{parskip} % line between paragraphs
\usepackage{pgfornament}
\usepackage{booktabs}
\usetikzlibrary{shapes.geometric,calc}
\usepackage{subfiles}
% suits


\newcommand{\BC}{\textcolor{OliveGreen}{$\clubsuit$}}
\newcommand{\BD}{\textcolor{RedOrange}{$\vardiamondsuit$}}
\newcommand{\BH}{\textcolor{Red2}{$\varheartsuit${}}}
\newcommand{\BS}{\textcolor{MidnightBlue}{$\spadesuit${}}}

%suits for pdf-friendly titles
\newcommand{\pdfc}{\texorpdfstring{\BC{}}{C}}
\newcommand{\pdfd}{\texorpdfstring{\BD{}}{D}}
\newcommand{\pdfh}{\texorpdfstring{\BH{}}{H}}
\newcommand{\pdfs}{\texorpdfstring{\BS{}}{S}}

\newcommand*\numcircledtikz[1]{\tikz[baseline=(char.base)]{
            \node[shape=circle,draw,inner sep=1.2pt] (char) {#1};}} 

%Titlepage
\addto\captionsitalian{
    \renewcommand{\chaptername}{Lezione}
      \renewcommand*{\chapterautorefname}{Lezione}%
}
\addto\extrasitalian{
      \renewcommand*{\chapterautorefname}{Lezione}%
}
\newcommand\SA{{\smaller{SA}}\xspace}
\newcommand\pic{Picche\xspace}
\newcommand\cu{Cuori\xspace}
\newcommand\qu{Quadri\xspace}
\newcommand\fio{Fiori\xspace}

\newenvironment{bidtable}
{\begin{tabbing}

    xxxxxx\=xxxxxx\=xxxxxx\=xxxxxx\=xxxxxx\=xxxxxx\=xxxxxx\=xxxxxx\=xxxxxx\=xxxxxx\=\kill}
{\end{tabbing} }%

\newenvironment{sviluppi}
{\begin{tcolorbox}[colframe=azzurro,title=Sviluppi particolari]}
    {
\end{tcolorbox} }%

\newcounter{attenzionectr}
\newcounter{regolactr}
\newenvironment{attenzione}[1]
{
    \refstepcounter{attenzionectr}
\begin{tcolorbox}[colframe=red!80!white,title=#1]}
    {
\end{tcolorbox} }%
\newtcolorbox[auto counter, list inside=regole]{regola}[2][]{%
colframe=green!40!black,
title={Regola \thetcbcounter: #2}
}
\AtBeginEnvironment{regola}{\medskip}
\AtEndEnvironment{regola}{\medskip}
%\newenvironment{regola}[1]
%{
%    \refstepcounter{regolactr}
%    \medskip
%    \begin{tcolorbox}}
%    {
%\end{tcolorbox}
%    \medskip
% }%
\newcommand{\regolactrautorefname}{Regola}
% \newenvironment{bidding}%
% {\begin{tabbing}
% xxxxxx\=xxxxxx\=xxxxxx\=xxxxxx \kill
% }{\end{tabbing} }%end bidding

% writing hands
\newcommand{\cards}[1]{\textsf{#1}}
%\newcommand{\spades}[1]{\BS\cards{#1}}
%\newcommand{\hearts}[1]{\BH\cards{#1}}
%\newcommand{\diamonds}[1]{\BD\cards{#1}}
%\newcommand{\clubs}[1]{\BC\cards{#1}}
\newcommand{\void}{--}
\newcommand{\vhand}[4]{\spades{#1}\\\hearts{#2}\\\diamonds{#3}\\\clubs{#4}}

% The \Box should always appear the same distance from the left margin
\newcommand\onesuit[4]%
{%
    \begin{center}%
        \begin{tabular}{>{\hfill}p{3cm}cp{3cm}}
            & \cards{#2} \\%
            \cards{#1}& $\Box$    & \cards{#3} \\%
                      & \cards{#4} %
        \end{tabular}
    \end{center}%
}

% A special command if the south hand is not shown to avoid whitespace
\newcommand\onesuitenw[3]%
{%
    \begin{center}%
        \begin{tabular}{>{\hfill}p{3cm}cp{3cm}}%
                & \cards{#2} \\%
            \cards{#1}& $\Box$    & \cards{#3}%
        \end{tabular}%
    \end{center}%
}

\newcommand\dealdiagram[5]%
{%
    \begin{center}%
        \begin{tabular}{>{\hfill}p{3cm}cp{3cm}}
            \pbox{20cm}{\small #5}& \pbox{20cm}{#2} \\%
            \pbox{20cm}{#1}& $\Box$    & \pbox{20cm}{#3} \\%
                           & \pbox{20cm}{#4} %
        \end{tabular}
    \end{center}%
}

\newcommand\dealdiagramenw[4]%
{%
    \begin{center}%
        \begin{tabular}{>{\hfill}p{3cm}cp{3cm}}
            \pbox{20cm}{\small #4}& \pbox{20cm}{#2} \\%
            \pbox{20cm}{#1}& $\Box$    & \pbox{20cm}{#3} \\%
        \end{tabular}
    \end{center}%
}

\newcommand\dealdiagramew[2]%
{%
    \begin{center}%
        \begin{tabular}{>{\hfill}p{3cm}cp{3cm}}
            \pbox{20cm}{#1}& $\Box$    & \pbox{20cm}{#2} \\%
        \end{tabular}
    \end{center}%
}

%\newcommand*{\plogo}{\includegraphics[scale=.5, trim={1cm 1cm 1cm 1cm}, clip]{quote associative 2022.jpg}} % Generic dummy publisher logo
\newtcbox{\miamano}{colback=white, colframe=white, arc=0pt,outer arc=0pt}
\renewcommand*\familydefault{\sfdefault}
\begin{document}
\definecolor{green}{named}{OliveGreen}
\definecolor{orange}{named}{RedOrange}
\definecolor{red}{named}{Red2}
\definecolor{blue}{named}{MidnightBlue}
\setdefaults{colors=4A}
\gamefont{\sffamily\bfseries\large}
\setdefaults{compmid=\boardtext,bidfirst=N}
\section{Board 1}
\newgame
\boardnr{1}
\northhand{AQ8}{743}{A743}{AK4}
\southhand{742}{KQJ6}{K2}{8653}
\westhand{KJT63}{T2}{QJT8}{72}
    \easthand{95}{A985}{965}{QJT9}
    \begin{bidding}-
        1N & p & 2C & p\\
        2D & p & 3N & ---\\
    \end{bidding}

\showAll*+

\paragraph{Dichiarazione:}
Nord con 17 punti bilanciati apre di 1\SA. Est passa. Sud, con 9 punti e una quarta maggiore va alla ricerca del fit
attraverso la dichiarazione di 2\Cl Stayman. Nord, non possedendo 4e nobili, risponde 2\Di (si vedano dispense per le
risposte). Una volta assicurata la mancanza di fit in un colore maggiore, Sud chiude a manche.

\paragraph{Gioco:}
Nord conta 4 vincenti. Se in grado di fare 3\He e 2\Sp arriva a 9. Si tratta, quindi, di giocare 3 volte verso le cuori
del morto mantenendo l'ingresso con il K\Di per godere della 4a cuori. Quindi:

\begin{play}*{E}
    CQ & 3 & 2 & K\\
    H3 & 5 & J & 2\\
    S2 & T & Q & 5\\
    H4 & 8 & Q & T\\
    D2 & T & A & 5\\
    H7 & A & 6 & S3\\
    CJ & 5 & 7 & A\\
    D3 & 6 & K & 8\\
    HK & S6 & S8 &9\\
    S4 & J & A &9\\
\end{play}

Contratto mantenuto.
\newpage

\section{Board 2}

\newgame
\boardnr{2}
\northhand{QJT63}{A62}{K6}{QJ3}
\southhand{K952}{853}{AQ5}{K65}
\westhand{87}{T97}{T732}{AT72}
\easthand{A4}{KQJ4}{J984}{984}
\begin{bidding}-
        - & p & 1C & p\\
        1S & X & 2S & p\\
        4S & ---\\
    \end{bidding}

\showAll*+

\paragraph{Dichiarazione:} Sud apre con 12, Nord dichiara 1\Sp. Est, essendo passato di mano, può dichiarare contro (non può avere
12+ punti!), mostrando 4\Di e 4\He. Sud dichiara 2\Sp: anche se si tratta di una dichiarazione libera, abbiamo detto che
il fit è sacro, ovvero che è sempre necessario mostrarlo, anche con mano minima.

\paragraph{Gioco:}
Dopo l'attacco K\He, possediamo 10 vincenti: 4\Sp, 1\He, 3\Di, 2\Cl. Nonostante questo, dobbiamo ricordarci che
è necessario che gli avversari non facciano 4 prese prima che noi ne facciamo 10! Se, infatti, battiamo atout, gli
avversari potranno vincere 1 presa a \pic, 2 a \cu e 1 a \fio, per un totale di 4. Si tratta, quindi, di giocare 3 volte
\qu al fine di scartare una cuori ed esercitare il potere di controllo delle atout in questo seme.

\begin{play}*{E}
    HK & 3 & 7 & A\\
    DK & 4 & 5 & 2\\
    D6 & 8 & Q & 3 \\
    DA & 7 & H2 & D9\\
\end{play}

Ora Est-Ovest non possono fare nulla per battere il contratto se Nord batte atout.

\newpage

\section{Board 3}
\newgame
\boardnr{3}
    \northhand{KJT974} {AT3} {Q43} {9}
    \westhand{53} {Q865} {JT9} {AT52}
    \easthand{AQ6} {K4} {A865} {KQJ7}
    \southhand{82} {J972} {K72} {8643}
\begin{bidding}-
        - & - & p & p\\
        p & 1D & p & 1H\\
        1S & 1N & p & 3N\\
    \end{bidding}
\showAll*+

\paragraph{Dichiarazione:} Est apre, Ovest risponde 1\He e Nord, possedendo i requisiti per l'intervento a livello 1 (almeno
8 punti, almeno 5 carte nel colore da dichiarare, almeno un onore nel colore da dichiarare), dichiara 1\Sp. Est, che ora
  non è in una situazione di dichiarazione obbligata dal cambio di colore forzante del rispondente, dichiara 1\SA.
  Questa è una dichiarazione libera e, quindi, mostra:
  \begin{enumerate}
      \item Una mano particolarmente sbilanciata.
          \item Una mano forte.
  \end{enumerate}

  Può una dichiarazione di senz'atout mostrare una mano sbilanciata? No, perché questa preferirebbe dichiarare un
  colore. Deve, quindi, trattarsi di una mano forte: una mano di 18--20 punti bilanciata (non 15--17, con cui est
  avrebbea aperto 1\SA). A questo punto, Ovest, sapendo di essere oltre il giustificativo di punteggio per la manche
  (24--25 punti) dichiara 3\SA: non può, infatti, esserci il fit a \cu, con il quale Est avrebbe dichiarato 4\He.

\paragraph{Gioco:} Possedendo, dopo l'attacco \pic (il colore di Nord), 6 prese, si tratta di affrancarne altre 3. Il
colore più promettente sono le \qu, con una manovra di impasse alla Q e al K. Ma attenzione! Per questa manovra ci
servono due ingressi al morto, in quanto andrà ripetuta! Importante non incassare troppe \fio, in quanto potreste rimanere incastrati al morto (provate a giocare due volte fiori e poi mangiarvi il J con
l'Asso, e sull'onore di \qu coprire con la Q: rimarreste bloccati al morto, costretti a cedere una presa a Nord, che
possiede le \pic franche dopo che Sud è tornato \pic in presa con il K\Di).

\begin{play}*{S}
    S8 & 3 & 9 & Q\\
    CK & 3 & A & 9\\
    DJ & 3 & 5 & K\\
    S2 & 5 & T & A\\
    C7 & 4 & T & H3\\
    DT & Q & A & 2\\
    D6 & 7 & 9 & 4\\
    C2 & HT & CQ & 6\\
\end{play}

\section{Board 4}
\newgame
\boardnr{4}
\northhand{432}{JT97}{AK98}{52}
\southhand{85}{Q8}{T7432}{QT97}
\westhand{KQJT6}{5432}{Q5}{K8}
\easthand{A97}{AK6}{J6}{AJ643}
\begin{bidding}-
    -& -& - & p\\
    1N & p & 2C & p\\
    2D & p & 2S & p\\
    4S & ---\\
\end{bidding}

\showAll*+

\paragraph{Dichiarazione:} Est apre di 1\SA e Ovest, possedendo una quinta maggiore in una mano con forza sufficiente
a giocare manche (15 + 11 > 24--25 punti), dichiara 2\Cl Stayman per poi dichiarare 2\Sp, mostrando 5\Sp in mano da
manche. Est, notando che Ovest è passato, sa che non può esserci Slam e chiude a manche.

\paragraph{Gioco:} Tante le linee possibili. Tra queste:
\begin{enumerate}
    \item Giocare tre giri di \cu, per poi tagliare il 4\kern-.1em° con il 9\Sp.
    \item Tagliare due giri di \fio ed effettuare un affrancamento di lunga, mantenevo gli ingressi al morto.
\end{enumerate}
Entrambe questo linee hanno successo.

\section{Board 5}

\newgame
\boardnr{5}
    \northhand{T3} {KT82} {KQJ2} {AJT}
    \westhand{874} {J93} {4} {Q76532}
    \easthand{AKJ62} {76} {A953} {98}
    \southhand{Q95} {AQ54} {T876} {K4}

    \begin{bidding}-
        1D & 1S & X & p\\
        2H & p & 3H & p\\
        4H & ---\\
    \end{bidding}
    \showAll*+

\paragraph{Dichiarazione:} Nord apre, Est entra con 1\Sp possedendo tutti i requisiti per l'intervento a livello 1 (5+
carte, colore solido e anche punti). Sud, pur avendo l'appoggio a \qu,
è maggiormente interessato al fit a \cu. Non avendo abbastanza punti e lunghezza per dichiararle a livello 2, sceglie
una dichiarazione che mostra la ricerca di un fit 4-4 e un po' di punti (almeno 8): Contro. Nord, che, ricordiamoci, è in
una situazione di dichiarazione obbligata, è costretto a dichiarare la quarta di \cu, anche se supera il livello di
guardia: infatti, il fit a cuori è dato per assunto dopo che il compagno contra. Sud, a questo punto, appurato il fit
a cuori, dichiara un \emph{parziale non minimo} al fine di mostrare una mano invitante di 10--11 punti onori. Nord,
possedendo un apertura minima nel massimo del range (14 punti onori), dichiara 4\He.

\paragraph{Gioco:}
Est attacca di Asso di \pic. A questo punto Est, realizzando che il dichiarante ha aperto 1\Di (ricordate la licita!) e che
quindi ne possiede almeno 4, sa che il compagno ha il singolo di \qu. Si tratta quindi di dargli il
taglio prima di incassare il Re di picche, che permette di fornire un secondo taglio al compagno. Si veda, di seguito,
il corretto ordine di gioco.

\begin{play}*{E}[H]
    SA & 5 & 4 & 3\\
    DA & 6 & 4 & 2\\
    D9 & 7 & H3 & DJ\\
    S8 & T & K & 9\\
    D5 & 8 & H9 & Q\\
\end{play}

Si noti, per il futuro, che il ritorno di Est con il 9 di \qu, essendo una carta alta, suggeriva al compagno di giocare
il seme più alto, ovvero \pic.

\section{Board 6}

\newgame
\boardnr{6}
    \northhand{AT92} {Q872} {A9} {A75}
    \southhand{K8} {J96} {T6432} {832}
    \westhand{J543} {AK54} {KQ85} {J}
    \easthand{Q76} {T3} {J7} {KQT964}
    \begin{bidding}-
        - & p & p & 1D\\
        X & 2C & --- \\
    \end{bidding}
\showAll*+

\paragraph{Dichiarazione:}Nonostante molti di voi abbiano preferito, con le carte di Nord, dichiarare 1\Sp, la
dichiarazione corretta è quella di Contro: ricordate, infatti, che l'intervento a colore richiede almeno 5 carte. Est
dichiara 2\Cl sul contro, promettendo 5+ carte e meno di 11 punti. Ovest non ha di meglio da dire e passa, come tutti
gli altri giocatori.

\paragraph{Gioco:} Un buon attacco è il K\Sp. Dopo 3 giri di \pic, di cui il terzo tagliato da Sud, si tratta, per Est,
semplicemente di battere atout e incassare 5\Cl, 2\He e 1\Di.

\newpage
\section{Board 7}
\newgame
\boardnr{7}
    \northhand{8643} {93} {QJ6} {T942}
    \westhand{AJT9} {KQJ5}{32} {A87} 
    \easthand{Q52} {AT864} {K98} {J5}
    \southhand{K7} {72} {AT754} {KQ63}
    \begin{bidding}-
        - & - & 1D & X\\
        p & 2H & p & 4H\\
    \end{bidding}

\showAll*+

\paragraph{Dichiarazione:} Sud apre. Est, possedendo 15 punti e una mano bilanciata, ma non il fermo a \qu,
è impossibilitato a dichiarare 1\SA. Dichiara quindi contro, avendo almeno la 4-3 tra \cu e \pic e anche 3 \fio, quindi
giocabilità negli altri semi. Est,
possedendo 10 punti e la quinta di \cu, possiede una mano invitante: è, infatti, desideroso di dichiarare manche se il
compagno ha un contro non minimo (almeno 14 punti). Dichiara, quindi, un parziale non minimo (2\He), con significato
invitante. Ovest, possedendo un buon Contro (15 punti e 4 carte di fit contro le sole tre attese dal compagno), dichiara
4\He.

\paragraph{Gioco:} Dopo l'attacco K\Cl, si tratta di battere le atout in due giri finendo in Mano, per poi giocare la
Q\Sp al fine di effettuare l'impasse: 5\He, 4\Sp, 1\Cl, ovvero 10 prese.

\section{Board 8}

\newgame
\boardnr{8}
    \northhand{8532} {876} {63} {AT72}
    \southhand{AJ76} {QJT5} {87} {KQ8}
    \easthand{K} {AK42} {AKT942} {94}
    \westhand{QT94} {93} {QJ5} {J653}
    \begin{bidding}-
        - & - & - & p\\
        p & 1D & X & p\\
        1S & 2H & p & 3D\\
    \end{bidding}
    \showAll*+

    \paragraph{Dichiarazione:} Est, possedendo mano di Rovescio, apre 1\Di. Sud non possedendo un colore quinto, con buona giocabilità
    negli altri colori e con 12+ punti, dichiara Contro. Ovest, con il minimo, passa. Nord, essendo costretto
    a dichiarare (ricordate che il Contro vi chiede di dichiarare un contratto di altro tipo, con preferenza di uno dei
    tre semi alternativi a quello di apertura) e possedendo 4\Sp e pochi punti,preferisce dichiarare 1\Sp, ovvero il
    contratto che vuole giocare: è infatti alta la possibilità di fit, e giocando in altri semi la mano di Nord farà
    certamente meno prese: possiamo, infatti, sperare di fare qualche taglio con le \pic.

    Est, possedendo una bicolore (due semi) di 10 carte, per giunta non minima,
    è giustificato a dichiarare. 2\He è la dichiarazione migliore: il compagno saprà che abbiamo almeno 5\Di (niente ci
    impedisce, come in questo caso, di averne 6) e 4\He, permettendogli di scegliere il contratto migliore. Ricordate
    sempre questa massima quando si tratta di dichiarazione: cercate di rendere il compagno il \emph{boss}, ovvero di
    fornirgli tutte le informazioni necessarie al fine di compiere la migliore dichiarazione. Un'ottima dichiarazione
    è quella che permette al compagno di compiere una decisione complessa con il minimo sforzo. In questo caso, fornigli
    informazioni su 2 colori (invece che solo su 1, dichiarando 2\Di, per altro nascondendo la nostra forza) è la cosa migliore.

    \paragraph{Gioco:} Niente da dire, si tratta semplicemente di affrancare la Q\Sp per arrivare a 9 prese, oppure di
    tagliare una \cu con il medesimo risultato.

    \section{Board 9}
\newgame
\boardnr{9}
    \northhand  {AQ6}   {KJ52}  {763}  {AKQ}     
    \westhand   {KJT92}   {Q3}  {QJT}   {JT3}    
    \easthand   {754} {AT64}    {98}  {9764}    
    \southhand  {83}   {987}  {AK542}{852}       

    \begin{bidding}-
        1C & p & 1D & 1S\\
        1N & p & 3N &---\\
    \end{bidding}

\showAll*+

\paragraph{Dichiarazione:} Nord, dopo l'intervento di 1\Sp, non è più obbligato a parlare (la parola, infatti, tornerà
a Sud, come da sua richiesta effettuata tramite la dichiarazione di un nuovo colore forzante, ovvero 1\Di). La sua dichiarazione è,
quindi, \emph{libera}. Le dichiarazioni libere denotano:
\begin{itemize}
    \item Mani particolarmente sbilanciate, in possesso di 6+ belle carte nel seme di apertura o di almeno 10 carte tra
        i due semi dichiarati.
        \item Mani particolarmente forti.
\end{itemize}

In questo caso vale il discorso fatto per il Board 3: dichiarando i \SA neghiamo una mano bilanciata, di conseguenza
garantendo una mano non minima. Non possiamo possedere 15--17 bilanciati in quanto avremmo aperto 1\SA: rimane, quindi,
solo la fascia 18--20. Sud, a questo punto, chiude a manche.

\paragraph{Gioco:} Dopo l'attacco \pic, per nord si tratta di trovare due prese. Il colore più promettente sono le
\qu, in cui possediamo otto carte, ed è quindi probabile che i resti avversari siano 3-2. Ma attenzione! La mano di Sud,
povera di punti, è povera anche di rientri! Se, provando ad effettuare un affrancamento di lunga, giochiamo A,
K e piccola a \qu, ci troviamo con il colore di quadri sì affrancato, ma \emph{irraggiungibile}. Si tratta, quindi,
sapendo di dover cedere una presa, di cederla immediatamente, al fine di mantenere il collegamento (comunicazione
interna al colore, si vedano le pagine 13 e 27--28 delle dispense, con particolare attenzione alla Regola 13) nel colore
di \qu. Il gioco deve, quindi, svolgersi nella seguente maniera:

\begin{play}*{E}
    S4 & 3 & 9 & Q\\
    D3 & 9 & 2! & T\\
    ST & A & 5 & 8\\
    D6 & 8 & K & J\\
    DA & Q & 7 & H4\\
\end{play}

A questo punto il colore di \qu è \emph{affrancato} e noi ci troviamo al Morto, dove possiamo usufruirne.


\section{Board 10}

\newgame
\boardnr{10}
    \northhand{A} {KQ8754} {KT2} {Q76}
    \westhand{J865} {T} {AJ} {KJT542}
    \easthand{97432} {J93} {Q643} {3}
    \southhand{KQT} {A62} {9875} {A98}

    \begin{bidding}-
        - & - & 1D & 2C\\
        2H & p & 3H & p\\
        4H & ---\\
    \end{bidding}
    \showAll*+

    \paragraph{Dichiarazione:} Ovest interviene di 2\Cl possedendo un bel colore sesto e più di 10 punti. Nord, con 14
    punti e una sesta, supera di gran lunga il requisito per una dichiarazione di 2 in competizione (10+ punti e almeno
    5 carte). Sud mostra il proprio appoggio minimo dichiarando 3\He. Nord, sapendo di possedere punteggio superiore al
      giustificativo di manche, dichiara 4\He.

      \paragraph{Gioco:} Fondamentale prendere di Asso l'attacco di 3\Cl: si tratta, infatti, certamente di un singolo
      data la dichiarazione di Ovest (se lasciato correre, rischiamo che Ovest rigiochi \fio, dando il taglio al
      compagno, che poi giocherà \qu per farsene dare un altro, facendoci perdere 4 prese: due tagli a \fio, il K\Cl
      e l'A di \Di). Sbloccare, successivamente, l'Asso di \pic, per poi battere le atout finendo al
      morto, dove giocheremo K e Q di \pic per scartare due quadri e giocare \fio verso la Q, un expasse della cui
      riuscita siamo sicuri dato l'intervento.

      \begin{play}*{E}[H]
          C3 & A & 2 & 6\\
          ST & 5 & A & 2\\
          HK & 3 & 2 & T\\
          HQ & 9 & 6 & C4\\
          H4 & J & A & C5\\
          SQ & 6 & D2 & S3\\
          SK & 8 & DT & S4\\
      \end{play}

\section{Board 11}
\newgame
\boardnr{11}
    \southhand{K542} {QJT9} {J3} {AJ5}
    \northhand{87} {86} {T8642} {Q864}
    \westhand{AQJT} {A743} {Q95} {32}
    \easthand{963} {K52} {AK7} {KT97}

\begin{bidding}-
    - & - & 1C & x\\
    p & 3N & ---\\
\end{bidding}

\showAll*+

\paragraph{Dichiarazione:} Sud apre e Ovest, possedendo l'apertura ma nessun colore quinto, ed avendo almeno la 4-3 nei
nobili, dichiara Contro. Est, in possesso di 13 punti e fermo nel colore avversario sa due cose: non c'è fit in un
colore maggiore (il compagno non può averne 5), e che il giustificativo per la manche è raggiunto. Dichiara, quindi,
3\SA.

\paragraph{Gioco:} Attacco di Q\He. Possiamo contare 6 vincenti. Le \pic offrono la possibilità di sviluppare le tre
ulteriori, tramite una manovra ripetuta di impasse. In presa con il K\He, quindi, giocheremo immediatamente \pic, per
tornare in Mano con l'Asso di \qu, ripetere l'impasse, tornare in mano con il k\Di e ripetere, per l'ultima volta,
l'impasse al K\Sp. Potenzialmente, con una manovra molto particolare e controintuitiva, una decima presa spetta al
dichiarante; riuscite a trovarla? Poco sotto il
diagramma della situazione a questo punto.

\begin{play}*{S}
    HQ & 3 & 6 & K\\
    S3 & 2 & T & 7\\
    D5 & 2 & A & 3\\
    S6 & 4 & J & 8\\
    D9 & 4 & K & J\\
    S9 & 5 & Q & D6\\
    SA & D8 & C7 & SK\\
\end{play}

\newgame
\boardnr{11}
    \southhand{-} {JT9} {-} {AJ5}
    \northhand{-} {8} {T} {Q864}
    \westhand{-} {A74} {Q} {32}
    \easthand{-} {52} {7} {KT9}

\showAll*

\vspace{11cm}
A questo punto, se Est gioca la Q\Di e Asso di \cu e \cu, Sud sarà costretto a prendere e regalare il K\Cl, dovendo
giocare dall'Asso di \fio. Di seguito la situazione, dopo  aver giocato la Q\Di e due giri di \cu, di cui il secondo preso da Sud con il
\Ten\He. Sud in presa:

\newgame
\boardnr{11}
    \southhand{-} {J} {-} {AJ}
    \northhand{-} {-} {-} {Q86}
    \westhand{-} {7} {-} {32}
    \easthand{-} {-} {-} {KT9}

\showAll*

\section{Board 12}
\newgame
\boardnr{12}
    \northhand{KQ64} {A65} {AKJ73} {4}
    \westhand{982} {T32} {952} {KT92}
    \easthand{A7} {KQJ74} {T4} {Q863}
    \southhand{JT53} {98} {Q86} {AJ75}
\begin{bidding}-
    - & - & - & p\\
    1D & 1H & X & p\\
    2S & p & 4S & ---\\
\end{bidding}

    \showAll*+

    \paragraph{Dichiarazione:} Nord, con 17 punti in mano sbilanciata, apre di 1\Di. Est, possedendo un bel colore di
    \cu quinto e l'apertura, dichiara 1\He. Sud vuole andare alla ricerca di un fit 4-4 negli altri colori: dichiara
    quindi Contro, in quanto possiede anche il minimo del punteggio richiesto per questa dichiarazione da parte del
    Rispondente: 8 punti. Nord, dal canto suo, possiede la quarta di \pic. Ma se la dichiarasse a livello 1 non
    supererebbe il livello di guardia (2\Di), mostrando, quindi, mano di dritto. \`E allora necessario dichiarare
    2\Sp, superando il livello di guardia e mostrando una mano di rovescio, sbilanciata con 4\Sp. Sud, sapendo che il
    giustificativo per la manche è raggiunto, dichiara 4\Sp.

    \paragraph{Gioco:} Dopo l'attacco di K\He, Nord può contare 9 vincenti (ricordate che le affrancabili in atout si
    contano sempre come vincenti). Si tratta, quindi, di trovare la decima presa: questa verrà necessariamente da un
    taglio, effettuato dopo la battuta delle atout, o della \cu al morto, o della \fio in mano.


    \section{Board 13}
\newgame
\boardnr{13}
    \northhand{Q9753} {KQ} {A86} {K86}
    \southhand{86} {AJT63} {T32} {T94}
    \westhand{AJ} {8752} {KQJ54} {53}
    \easthand{KT42} {94} {97} {AQJ72}

    \begin{bidding}-
        1S & p & 1N & 
        ---\\
    \end{bidding}

    \showAll*+

\paragraph{Dichiarazione:} Né Est né Ovest possiedono i requisiti per un intervento a livello 2 (6 carte almeno
discrete), di conseguenza entrambi passano.

\paragraph{Gioco:} Dopo l'attacco K\Di, Sud dovrà incassare 5\He, mangiandosi il secondo onore con l'Asso (sblocco).
Proverà, in seguito, l'expasse al K di \fio, che fallisce. Risultato: 1\SA -1.

\section{Board 14}

\newgame
\boardnr{14}
    \northhand{JT92} {K} {AQ852} {T94}
    \southhand{543} {A73} {JT943} {52}
    \westhand{Q876} {982} {K6} {A763}
    \easthand{AK} {QJT654} {7} {KQJ8}

    \begin{bidding}-
        - & 1H & p & 2H\\
        p & 3C{!} & p & 4H\\
    \end{bidding}

    \showAll*+

\paragraph{Dichiarazione:} Una dichiarazione nuova, quella di Est, di cui non abbiamo parlato durante il corso. Dopo
l'appoggio di Ovest, Est sa che la linea Est-Ovest potrebbe superare il giustificativo di manche (se Ovest ha 8--10
punti). In cerca di informazioni, fa un ulteriore dichiarazione, mostrando un altro colore e più punti del previsto:
3\Cl. Si tratta di una dichiarazione invitante, fatta però dall'Apertore. Ovest, in possesso di 9 punti e di un secondo
fit a \fio, dichiara 4\He.


\paragraph{Gioco:} Dopo l'attacco J\Di, Est possiede 11 vincenti. Si tratta, quindi, di battere atout immediatamente,
per trasformarle da potenziali ad effettive. Notate come, in realtà, le prese siano 10: perderemo, infatti,
irrimediabilmente, una \qu e due \cu. State sempre attenti a queste situazioni in cui le prese che rischiate di perdere
sono in numero maggiore a $13 - \mathit{n^{\circ} Vincenti}$ : si tratta, spesso (non in questo caso), di
situazioni in cui si devono effettuare manovre particolari.

\end{document}
