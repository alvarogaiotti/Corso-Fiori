\documentclass[../corsofiori.tex]{subfiles}
\begin{document}
\chapter{Vincenti e affrancabili}
\label{chap:lezione1}
\setdefaults{colors=4A}

\`E fondamentale, nel Bridge, mantenere il Contratto (un offerta vincente effettuata darante la Dichiarazione, ovvero
l'asta iniziale, che sancisce un seme da usare come \emph{atout}, o la sua assenza, ed un numero minimo di prese da
vincere), ovvero fare un numero di prese almeno pari a quelle dichiarate.
A questo fine è fondamentale contare il numero di prese certe che si possiedono, in modo da pianificare le proprie
successive mosse in maniera strategica e ottimale.

Si dicono \textbf{vincenti o franche} o buone, quelle carte che, se giocate immediatamente, sono certe di aggiudicarsi una
presa.

Si dicono \textbf{affrancabili} quelle carte che potranno diventare vincenti dopo una opportuna manovra.
La seguente mano:

\hand*!{AK3}{KQJT}{QJ}{AQJ3}


\noindent possiede due vincenti a Picche, zero a Cuori, zero a Quadri e una a \fio. Per quanto riguarda le \textbf{affrancabili}, ne
possiede zero a \pic, tre a \cu (dopo aver ceduto l'Asso), zero a \qu e una a \fio (dopo aver ceduto il Re).

\paragraph{Carte equivalenti} Si dicono equivalenti tutte le carte di valore contiguo. Quindi, ad esempio, se un
Re è accompagnato da un Asso, allora abbiamo due vincenti, se abbiamo \cards{\textbf{J\Ten983}} abbiamo quattro equivalenti
e un'affrancabile.

\paragraph{Il Giocante}
Il Giocante è colui che si è aggiudicato un Contratto. Dato questo fatto, egli ha un obiettivo da raggiungere: il
numero di prese scommesse. Avendo un numero di prese minimo da ottenere, egli organizza ed effettua le proprie manovre
in base alle vincenti che possiede nelle sue 26 carte (ricordatevi, c'è anche il Morto!). Il fatto di vedere 26 carte
fornisce un grosso vantaggio rispetto ai difensori, che non vedendo la loro forza combinata, non sono in grado di fare
un conteggio preciso come quello del Giocante.

\begin{figure}[ht]
    \onesuitNS*{AQ2}{K83}
    \caption{Un colore di 3 vincenti, rappresentato tra Mano e Morto.}\label{fig:colore}
\end{figure}

In \autoref{fig:colore}, Sud, il Giocante, quando vede il Morto scopre che il Re è una vincente (come la Dama del Morto).

\paragraph{La prima cosa da fare: contare le vincenti}
Il Giocante, una volta che il Morto scopre le sue carte, considera i colori uno per uno, tra Mano e Morto (vedere
l'esempio in \autoref{fig:colore}, dove le vincenti vengono contate integrando le due mani), e conta le
vincenti all'interno di ciascun colore.

Per contare le vincenti in un colore si deve tenere conto di due informazioni:
\begin{itemize}
    \item Il numero di vincenti
    \item Il numero di carte nella mano in cui la lunghezza del colore è massima.
\end{itemize}

Il secondo punto è fondamentale in quanto rappresenta il numero di volte massimo che il colore potrà essere giocato.
Di conseguenza, vige sempre la seguente regola:

\begin{regola}{Regola sulle vincenti}
    \textsc{Il numero di prese realizzabili in un seme è pari o inferiore al numero di carte nel lato lungo}

\end{regola}
\bigskip


\begin{wraptable}[5]{l}[0cm]{6cm}
    \onesuitNS{JT9}{AKQ}\hfill\onesuitNS{T9}{AKQJ2}\hfill\onesuitNS{AKJ942}{QT3}\hfill
\end{wraptable}

Nel primo caso abbiamo sei vincenti, ma solo tre carte dal lato più lungo (in questo caso Mano e Morto possiedono la
medesima lunghezza), quindi, potendo giocare il seme solo tre volte, le
vincenti sono limitate a tre; nel secondo caso cinque; nel terzo
caso sei, pur con lo stesso numero di vincenti prima dell'aggiustamento dovuto alla lunghezza del seme.

\paragraph{Seconda operazione: individuare il o i colori in cui vi sono prese affrancabili}

L'obiettivo di entrambe le coppie è massimizzare il proprio potenziale in modo da fare il maggior numero di prese. Il
modo migliore di ottenere il più alto ( o il necessario) numero di prese è tentare di affrancare delle prese in più.

Per affrancare delle prese è necessario giocare ripetutamente nel colore delle affrancabili, partendo con le carte alte
equivalenti possedute, spesso iniziando con quelle nella mano con meno carte nel seme selezionato\sidefootnote{Questo
processo di dare precedenza  alle carte equivalenti della mano più corta si schiama \emph{sblocco}.}.

Per ottenere il numero di prese ottenibili da un seme contenente affrancabili è necessario:
\begin{itemize}
    \item Contare il numero di equivalenti
    \item Sottrarvi il numero di vincenti avversarie in quel colore
\end{itemize}

Ricordate! Vige sempre la seguente regola:

\begin{regola}{Regola sulle affrancabili}
    \textsc{Il numero massimo di prese affrancabili è sempre limitato dal numero di carte del lato più lungo}
\end{regola}
\bigskip

\begin{minipage}[h][3cm][t]{.23\linewidth}
    \onesuitNS{JT3}{KQ92}


    4-1 = 3 prese
\end{minipage}\hfill
\begin{minipage}[h][3cm][t]{.23\linewidth}
    \onesuitNS{JT3}{KQ952}


    5-1 = 4 prese
\end{minipage}\hfill
\begin{minipage}[h][3cm][t]{.23\linewidth}
    \onesuitNS{JT32}{Q9854}


    5-2 = 3 prese
\end{minipage}\hfill
\begin{minipage}[h][3cm][t]{.23\linewidth}
    \onesuitNS{QT2}{KJ9863}


    6-1 = 5 prese
\end{minipage}

Ricordatevi, inoltre, che il primo colore da muovere (giocare) è quello che contiene più affrancabili.



\paragraph{La corsa all'affrancamento} Quando si gioca a Senza Atout (\SA o \small{NT}\normalsize) l'obiettivo di
entrambe le coppie è l'affrancamento dei propri semi. Di conseguenza è di fondamentale importanza non gettarsi a giocare
le proprie vincenti (detto anche \emph{incassare} le vincenti), bensì immediatamente mettersi all'opera per affrancare
il proprio colore\sidefootnote{Seme e colore sono sinonimi nel Bridge}. Questo perché, incassando le nostre vincenti,
corriamo il rischio di affrancare il seme degli avversari, il cui obiettivo è il medesimo del giocante: rendere i propri semi \emph{franchi},
giocando ripetutamente in quel colore.

    \onesuitAll*{AK}{74}{Q52}{JT9863}



    Se incassiamo l'Asso e il Re, rendiamo immediatamente franche le quattro affrancabili avversarie. Ora agli avversari
    basterà vincere una sola presa per poter ottenere cinque prese: una presa in un seme esterno, grazie alla quale
    potranno incassare le altre quattro nel seme in figura, che sarà ormai affrancato.
   Se, invece, ci asteniamo dal giocare l'Asso e il Re, gli avversari avranno bisogno di prendere ben tre volte in altri
   semi, al fine di affrancare e incassare questo seme: due prese per farci giocare l'Asso e il Re, e una per usufruire
   del colore.

   Vediamo, ora, una mano e la serie di passi che ci portano a decidere il modo migliore in cui giocarla.

    \newpage


\northhand{AJ9}{76}{K65}{QT932}
\southhand{KT2}{KQJ}{AJ3}{J84}


\begin{wraptable}[11]{l}[0cm]{6cm}
    \vspace{-.5cm}
    \showNS*
\end{wraptable}

\noindent L'attacco di Ovest è \Ten\Di.

\noindent Sud conta le vincenti in ogni colore:

\Sp: 3

\He: 0

\Di: 3 (il Fante ora è vincente!)

\Cl: 0\\
\noindent Affrancabili:

\Sp: 0

\He: 2

\Di: 0

\Cl: 3

Le \fio sono la sorgente di prese più promettente e, quindi, il colore da giocare per primo. Dopo aver giocato
\fio
due volte, avversari permettendo, si giocherà \cu.
\newgame


\paragraph{I punteggi dei contratti}

Ogni Contratto dichiarato ha, associato un punteggio, che viene assegnato quando il contratto viene almeno mantenuto. Il
punteggio è semplicissimo da calcolare:

\begin{itemize}
    \item \SA: 40 punti la prima presa, 30 le restanti
    \item \Sp, \He: i semi \emph{maggiori}, ogni presa vale 30 punti
    \item \Di, \Cl: i semi \emph{minori}, ogni presa vale 20 punti
\end{itemize}

Quindi il contratto di 3\He fornisce 90 punti ($30 \times 3$), quello di 4\Cl ($20 \times 4$).

Vi sono diversi tipi di contratto:

\begin{itemize}
    \item I contratti di manche, ovvero quelli che forniscono almeno 100 punti: 3\SA, 4\Sp, 4\He, 5\Di, 5\Cl
    \item I contratti parziali, ovvero quelli sotto il contratto di manche nel seme: 1/2\SA, 1-3\Sp, 1-3\He, 1-4\Di,
        1-4\Cl
    \item I contratti di slam, ovvero quelli che richiedono di fare almeno 12 prese. Si dividono in grande e piccolo,
        con il grande che richiede 13 prese, il piccolo 12. Questi contratti sono, quindi, quelli a
        \emph{livello}\sidefootnote{Con \emph{livello} si intende il numero che trovate nel cartellino della
        dichiarazione: il numero di prese che stiamo offrendo meno la base di prese per fare un'offerta, ovvero 6)} 6 o
        7.
\end{itemize}

\paragraph{I bonus (e i malus) dei contratti}

Ogni contratto ha dei bonus associati, che variano il base alla \emph{zona}, rappresentata dal colore riportato nel
Board\sidefootnote{Il Board è l'astuccio che contiene le carte già distribuite, in modo che sia agevole passarlo a
diversi tavoli mantenendo le stesse carte nelle stesse posizioni} per la vostra linea:

\bigskip
\begin{center}
\begin{tabular}{lcc}
    \toprule
    Contratto & Verdi (in prima) & Rossi (in zona)\\
    \midrule
    Parziale & 50 & 50\\
    Manche & 300 & 500\\
    Piccolo Slam & 500 & 750\\
    Grande Slam & 1000 & 1500\\
    \bottomrule
\end{tabular}
\end{center}
\bigskip

Il \emph{malus} per ogni presa in meno rispetto all'offerta fatta è di 50 in prima e 100 in zona. Ricordatevi che, non
avendo mantenuto il contratto, non vi viene assegnato nessun punteggio positivo.


\newpage
\section*{Esercizi}

\bigskip
\begin{tabular}{ccc}

    \begin{minipage}[h][3cm][t]{.30\linewidth}
    \onesuitNS*{AQ2}{K83}

    \begin{center}

        Vincenti: \rule{1cm}{.4pt}
    \end{center}
\end{minipage}
&
    \begin{minipage}[h][3cm][t]{.30\linewidth}
    \onesuitNS*{AQ2}{KJ83}

    \begin{center}
        Vincenti: \rule{1cm}{.4pt}
    \end{center}
\end{minipage}
&
    \begin{minipage}[h][3cm][t]{.30\linewidth}
    \onesuitNS*{832}{AKQJ}

    \begin{center}
        Vincenti: \rule{1cm}{.4pt}
    \end{center}
\end{minipage}
    \\[2cm]

    \begin{minipage}[h][3cm][t]{.3\linewidth}
    \onesuitNS*{JT9}{AKQ}

    \begin{center}
        Vincenti: \rule{1cm}{.4pt}
    \end{center}
\end{minipage}
    &
    \begin{minipage}[h][3cm][t]{.3\linewidth}
    \onesuitNS*{AKJ942}{QT3}

    \begin{center}
        Vincenti: \rule{1cm}{.4pt}
    \end{center}
\end{minipage}
    &
    \begin{minipage}[h][3cm][t]{.3\linewidth}
    \onesuitNS*{JT3}{AQ92}

    \begin{center}
        Vincenti: \rule{1cm}{.4pt}

        Affrancabili: \rule{1cm}{.4pt}
    \end{center}
\end{minipage}
    \\[2cm]
    \begin{minipage}[h][3cm][t]{.3\linewidth}
    \onesuitNS*{JT53}{Q9874}

    \begin{center}
        Affrancabili: \rule{1cm}{.4pt}
    \end{center}
\end{minipage}
&
\begin{minipage}[h][3cm][t]{.30\linewidth}
    \onesuitNS*{QT4}{KJ9863}

    \begin{center}
        Affrancabili: \rule{1cm}{.4pt}
    \end{center}
\end{minipage}
&
\begin{minipage}[h][3cm][t]{.3\linewidth}
    \onesuitNS*{JT53}{A9874}

    \begin{center}
        Affrancabili: \rule{1cm}{.4pt}
    \end{center}
\end{minipage}
\end{tabular}

\northhand{AQ8}{76}{KT932}{K65}
\southhand{K62}{A982}{QJ6}{A43}
\begin{center}
    \showNS*
        Vincenti \Sp: \rule{.3cm}{.4pt},\quad
        \He: \rule{.3cm}{.4pt}, \quad
        \Di: \rule{.3cm}{.4pt}, \quad
        \Cl: \rule{.3cm}{.4pt}\\

        Affrancabili \Sp: \rule{.3cm}{.4pt},\quad
        \He: \rule{.3cm}{.4pt},\quad
        \Di: \rule{.3cm}{.4pt},\quad
        \Cl: \rule{.3cm}{.4pt}
\end{center}

\end{document}
