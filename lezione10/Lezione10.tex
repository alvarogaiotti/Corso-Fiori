\documentclass[../corsofiori.tex]{subfiles}
\setcounter{regolactr}{36}
\setcounter{chapter}{9}
%\externaldocument[VIII-]{../lezione8/Lezione8}
\setdefaults{colors=4A}

\begin{document}
\chapter{Le ridichiarazioni dell'Apertore}
Il compito dell'Apertore, una volta che il compagno ha dichiarato, è riassumibile attraverso tre massime che egli deve
seguire:
\begin{itemize}
    \item Su risposte non forzanti: decide.
    \item Su risposte 1 su 1 descrive forza e distribuzione.
    \item Su risposte 2 su 1 descrive solo la distribuzione.
\end{itemize}

Descrivere la forza significa specificare con maggior precisione a quale intervallo di punteggio appartiene rispetto al
vasto range di punteggio che contiene l'apertura a livello 1 (12--20).
Questo range viene convenzionalmente diviso in due fasce di punteggio:
\begin{itemize}
    \item La fascia 12--15. Una mano appartenente a questa fascia viene definita \emph{di dritto}.
    \item La fascia 16--20. Una mano appartenente a questa fascia viene detta \emph{di rovescio}.
\end{itemize}

Il Rispondente, per avere un'idea più precisa di quale sia la forza di chi ha aperto, ha bisogno di sentire almeno due
dichiarazioni: la prima informa che l'Apertore ha 12--20 punti, la seconda specifica l'ulteriore fascia di punteggio,
ovvero se la mano è di dritto o di rovescio.

Diamo ora una definizione che ci servirà nella discussione successiva.

\begin{attenzione}{Livello di guardia}
    Si dice livello di guardia l'ultimo contratto superato il quale non vi è più certezza di conseguire un risultato
    ragionevole o positivo.
\end{attenzione}


\section{Ridichiarazioni dopo risposte 1 su 1}
Le risposte 1 su 1, come vi ricorderete, non hanno limite superiore, ma solo n limite inferiore (5 punti). L'Apertore ha
il compito di descriversi, ma anche di tenere presente che esiste un \emph{livello di guardia} oltre il quale la coppia,
se non trova fit e ha solo 18--19 punti in linea, potrebbe essere a rischio. Per questo motivo è necessario essere
prudenti e seguire la seguente regola.

\begin{regola}{Livello di guardia per l'Apertore}
    Dopo risposta 1 su 1, quando l'Apertore ha mano di dritto, se egli possiede una mano bilanciata il suo livello di
    guardia è 1\SA, se la mano è sbilanciata il suo livello di guardia è 2 \emph{nel colore di apertura}, ovvero il
    colore più lungo che possiede.
\end{regola}

La seconda dichiarazione dell'Apertore, quando egli possiede mano di dritto, non deve superare, quindi, il livello di
guardia.

\begin{regola}{Dichiarazioni che non superano il livello di guardia}
    Le seconde dichiarazioni dell'Apertore che non superano il livello di guardia mostrano mano di dritto
\end{regola}

\begin{center}
\hand!{AKJ7}{Q43}{42}{KJ86}\quad\begin{biddingpair}
    1C & 1D\\
    1S & \\
\end{biddingpair}
\end{center}

Sud ha mano di Diritto, bilanciata.
Il livello di Guardia è 1\SA; poiché la quarta di
Picche può essere descritta restando al di
sotto di tale livello, è corretto che Sud dica
1\Sp.

\begin{center}
\hand!{AK7} {QJ43} {42} {KJ86}\quad\begin{biddingpair}
    1C & 1S\\
    1\SA & \\
\end{biddingpair}
\end{center}

Sud ha mano di Diritto, bilanciata.
Il livello di Guardia è 1\SA; non
gli è
concesso dichiarare le Cuori, con cui
supererebbe il \ac{ldg}, quindi dichiarerà 1\SA.
Nessun timore di perdere fit: Nord non ha 4
carte di Cuori, oppure le ha, accanto a
Picche più lunghe: se è così sarà Nord a mostrarle al giro successivo.

\begin{center}
\hand!{AK72} {QJ43} {4} {KJ86}\quad\begin{biddingpair}
    1C & 1H\\
    2H & \\
\end{biddingpair}
\end{center}

La certezza del fit autorizza a superare l'\ac{ldg}; per comunicare fit Sud non ha nessuna dichiarazione più “bassa” di
2\He. Il fit , quando
è in un nobile, ha la precedenza assoluta: è
inutile che Sud dica 1\Sp, ha già “trovato” le
cuori.

\begin{center}
\hand!{J7} {KQ943} {AQ65} {J7}\quad \begin{biddingpair}
    1H & 1S\\
    2D & \\
\end{biddingpair}
\end{center}

Sud ha mano di Diritto, sbilanciata.
L'\ac{ldg} è 2\He; poiché le Quadri
possono essere descritte restando al di sotto
di tale livello, è corretto che Sud dica 2\Di.

\begin{regola}{Dichiarazioni che superano il livello di guardia}
Dopo risposta 1 su 1, una dichiarazione che supera il livello di guardia (che non sia un appoggio del colore dichiarato
dal Rispondente) mostra una mano di rovescio.
\end{regola}
\begin{center}
\hand!{AKJ7} {Q43} {4} {AKJ86}\quad\begin{biddingpair}
1C & 1D\\
3S & \\
\end{biddingpair}
\end{center}

Sud ha una mano di Rovescio, sbilanciata.
La quarta di Picche deve essere descritta
saltando un livello. Il compagno ne ricaverà
due informazioni:
\begin{enumerate}
\item Che l’Apertore ha almeno 16.
\item Che l’Apertore, avendo forza sufficiente per aprire 1\SA e non avendolo fatto, è di certo sbilanciato, quindi
    è lungo nel colore di apertura (le picche sono 4, altrimenti avrebbe aperto 1\Sp).
\end{enumerate}

\begin{center}
\hand!{A7} {AQ943} {4} {AKJ86}\quad\begin{biddingpair}
1H & 1S\\
3C & \\
\end{biddingpair}
\end{center}

Sud ha una mano di Rovescio, sbilanciata.
Le Fiori devono essere descritte saltando un
livello, perché se dichiarasse 2\Cl (entro il
LDG) il compagno dedurrebbe forza 12-15.


\begin{center}
\hand!{A987} {AQ3} {KQ874} {6}\quad\begin{biddingpair}
1D & 1S\\
3S& \\
\end{biddingpair}
\end{center}

Sud deve comunicare fit e forza;
appoggiando le Picche a livello 3 descrive
una mano di Rovescio minimo (circa 16-18);
Se tale è la forza, in automatico la mano è
sbilanciata, altrimenti avrebbe aperto 1\SA.
Quindi
il salto a 3\Cl ottiene anche di
mostrare un colore di Quadri lungo (oppure una tricolore). Con mano di Rovescio
massimo (19-20) Sud direbbe 4\Sp.

\begin{center}
\hand!{K9} {AQ43} {QJ42} {AK6}\quad\begin{biddingpair}
1D & 1S\\
2\SA \\
\end{biddingpair}
\end{center}

Sud ha troppo per aprire 1\SA (15-17) e
troppo poco per 2\SA (21-22). Per aprire si
comporta allo stesso modo della fascia 12-14
e poi, con il salto a 2\SA (che supera il LDG)
descrive al compagno la bilanciata
intermedia, di 18-20.


\section{Ridichiarazioni dopo risposte “due su uno”}

Sulle risposte 2 su 1, forzanti fino a manche, l’Apertore
descrive senza fare distinzioni tra diritto e rovescio.

\newgame
\westhand{93} {AK73} {KQJ5} {983}
\easthand{AJT} {62} {A73} {KQ754}
\begin{center}
\showAll
\quad\begin{biddingpair}
    1D & 2C\\
    2H & 3N\\
    p & \\
\end{biddingpair}
\end{center}

La manche più ragionevole dal
punto di vista di Est è 3\SA;
quando sente Ovest dichiarare il
solo colore che lo preoccupa,
non ha più problemi e propone
3\SA. Ovest, che ha mano di dritto, non ha niente da aggiungere.

\section{Dichiarazioni successive del Rispondente}

Se dopo che l’Apertore si è descritto con due dichiarazioni, il Rispondente è in grado di dichiarare un contratto
finale, lo fa. Se gli servono altre descrizioni, non avrà alcun problema se l'obiettivo Manche è già stato sancito da
una risposta 2 su 1: non potrà mai avvenire che il dialogo si interrompa prima della meta.

Ma se la prima risposta è stata 1 su 1, dovrà fare più attenzione: l'unico modo per garantirsi che l’Apertore parli
ancora è dichiarare un colore nuovo (che funziona da pungolo, e non richiede effettiva lunghezza).

\begin{regola}{Cambi di colore forzanti del Rispondente}
    Al primo giro come al secondo, il Rispondente che dichiari colori nuovi si garantisce che l’Apertore descriva
    ancora.
\end{regola}

Il \emph{cambio di colore} è dunque il motore con cui il Rispondente tiene aperto il dialogo. Esattamente come accade
per la prima risposta: un colore maggiore chiede di sé, un colore minore chiede una descrizione generica.

\newgame
\westhand {Q9} {KJ76} {K982} {A72}
\easthand  {KJ764} {AQ43} {Q3}{J8}
\showAll
\quad
\begin{biddingpair}
    1D & 1S\\
    1\SA & 2H\\
    3H & 4H\\
\end{biddingpair}
\begin{minipage}{0.3\textwidth}
\begin{tabular}{l}
1\Sp chiede se c’è fit di 4 carte\\
2\He chiede se c’è fit di 4 carte,
\\
o in alternativa se c’è fit\\
di 3 carte a
Picche.
\end{tabular}
\end{minipage}

Invece:

\newgame
\westhand {J82} {AJ95} {83} {AK52}
\easthand {KQ764} {K8} {AJ7} {874}
\showAll
\quad
\begin{biddingpair}
    1C & 1S\\
    1\SA & 2D\\
    3S & 4S\\
\end{biddingpair}\quad
\begin{minipage}{0.2\textwidth}
\begin{tabular}{l}
2\Di: chiede se c’è fit\\
di 3 carte a
Picche.\\
2\Sp: fit terzo
\end{tabular}
\end{minipage}

Se vi sembra astruso tutto questo (\emph{non sarebbe più semplice ripetere le
Picche?}) provate a considerare il caso in cui il Rispondente abbia sì la lunga nel
colore in cui ha risposto, ma voglia assolutamente giocare un parziale:
che altra dichiarazione avrebbe
Est, se non 2\Sp, per spiegare al
compagno che deve tacere e che
l’unico contratto possibile sulla
linea è proprio ed esattamente 2\Sp?

\begin{center}
\newgame
\westhand {82}   {AK95}{A983}{Q82}
\easthand {QJT654}{863} {K7}  {64}
\showAll
\quad
\begin{biddingpair}
    1D & 1S\\
    1\SA & 2S\\
\end{biddingpair}
\end{center}

\begin{regola}{Dichiarazioni del Rispondenten differenti dal cambio di colore}
Dopo risposta 1 su 1, le dichiarazioni a senz’atout oppure di colori vecchi (già dichiarati da
uno dei due) da parte del Rispondente sono \emph{non forzanti.}
\end{regola}

1\SA e tutti i colori vecchi detti al minimo livello escludono tassativamente manche.
2\SA e tutti i colori vecchi detti a salto sono invitanti: chiedono di passare col minimo
e rialzare col massimo.

\begin{wraptable}[5]{l}[0cm]{2.5cm}
\begin{biddingpair}
    1C & 1H\\
    1S & \ldots\\
\end{biddingpair}
\end{wraptable}

\mbox{}\\
\noindent
Da parte di Est sono scoraggianti: 1\SA, 2\Cl, 2\He, 2\Sp.\\
Sono invitanti a manche (circa 11 punti): 2\SA, 3\Cl, 3\He, 3\Sp.\\
L’unica dichiarazione forzante, in questa situazione, è 2\Di!

\end{document}
