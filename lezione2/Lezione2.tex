\documentclass[../corsofiori.tex]{subfiles}
\begin{document}
\setcounter{chapter}{1}
\setcounter{regolactr}{2}
\chapter{Il gioco in difesa}
\setdefaults{colors=4A}

L'obiettivo dei difensori è quello di battere il Contratto, ovvero fare un numero di prese almeno pari al complemento a
14 delle prese del contratto: se Sud sta giocando 1\SA, l'obiettivo dei difensori è $14 - 7 = 7$ (l'obiettivo di prese del
giocante).

\paragraph{L'attacco nei contratti a \SA}

L'attacco (ovvero la prima carta giocata durante la \emph{smazzata}\sidefootnote{Si dice \emph{smazzata} l'insieme  delle 52 carte distribuite tra i quattro giocatori, 13 per giocatore.} è un momento importante per raggiungere l'obiettivo di battere il
Contratto.
Anche per i difensori vale ciò che è stato detto durante la prima lezione: non bisogna avere fretta di incassare le
proprie vincenti. L'obiettivo è quello di affrancare delle prese, e le vincenti negli altri semi ci serviranno per poter
usufruire del seme affrancato. 
Con le seguenti carte:
\begin{center}
    \hand{A4}{94}{A762}{KQJ92}
\end{center}
il seme in cui attaccare è \fio. Se si inizia giocando i due Assi, gli avversari si troverebbero Re, Dame e Fanti
affrancati! Inoltre, a quel punto, anche riuscendo ad affrancare il seme di \fio, quando potremmo incassarlo? Non
riuscendo più a prendere, dovremmo sperare che l'avversario giochi \fio lui stesso, un regalo assai improbabile! Ricordate questa regola generale, che vale anche per il Giocante:

\medskip
\begin{attenzione}{Attenzione}

    Per incassare un seme affrancato è necessario possedere un \emph{ingresso}, ovvero una vincente in un altro seme, oppure
    una carta del seme affrancato posseduta dalla mano del compagno, che costituisce un
    \emph{collegamento}.

\end{attenzione}

\medskip

Un \emph{rientro} è esterno al colore, un \emph{collegamento} è, invece, interno.
    Questi termini valgono sia per i difensori, che per Giocante e Morto.

Quando dovete scegliere una carta con cui attaccare, seguite la seguente regola:

\medskip
\begin{regola}{Scelta dell'attacco}\label{regola:scelta dell'attacco}
    \textsc{Giocate una carta nel colore più lungo. Se avete due semi di pari lunghezza, scegliete quello con più onori}
\end{regola}

\medskip

\hand{JT4}{QJT3}{J762}{K2} : \cu

\hand{Q9732}{KQ}{AJ62}{92} : \pic

Come avrete potuto provare durante le vostre prime mani, è importante, tra difensori, avere un codice comune (un
linguaggio delle carte) al fine di comportarsi nella maniera più vantaggiosa.

Per impostare questo linguaggio, dividiamo innanzitutto le carte in due categorie: le sei di rango più elevato e le sette
rimanenti.


\begin{center}
    \textbf{Onori (e vice-Onori): A, K, Q, J, \Ten, 9\qquad e cartine: 8,7,6,5,4,3,2}
\end{center}

Diamoci, ora, due principi per la modalità di attacco.

\medskip
\begin{regola}{Attacco dall'alto}
    Quando si attacca con una carta della prima categoria si garantisce la presenza della carta immediatamente inferiore, e
    si nega quella immediatamente superiore:
\begin{tabular}{l l}
    \cards{\textbf{\underline{K}QJ\Ten3}} & il Re (nega l'Asso, promette la Dama)\\
    \cards{\textbf{\underline{Q}J93}} & la Dama (nega il Re, promette il Fante)\\
    \cards{\textbf{Q\underline{\Ten}976}} & il Dieci (nega il Fante ma non superiori, promette il 9)\\
\cards{\textbf{\underline{J}\Ten94}} & il Fante (nega la Dama  ma non superiori, promette il Dieci)\\
    \cards{\textbf{A\underline{J}\Ten943}} & il Fante (nega la Dama ma non superiori, promette il Dieci)

\end{tabular}
\end{regola}
\medskip

Quando si attacca, quindi, si gioca la \emph{testa} della sequenza. Ma attenzione: l'attacco di sequenza richiede un
seme solido. Oltre alla sequenza di due onori è necessario possedere un altro onore o, in alternativa, la carta
inferiore a quella che renderebbe la sequenza di tre onori: nel caso di una sequenza di Dama e Fante, il 9; nel caso di
una sequenza di Re e Dama, il \Ten.

Se nel colore sono presenti due sequenze, per facilitare l'affrancamento si dà precedenza a quella superiore:

\cards{\textbf{\underline{K}Q\Ten93}}\qquad\cards{\textbf{\underline{A}KJ\Ten3}}

\medskip
\begin{regola}{Attacco di cartina}
    Se non si può attaccare con la testa di una sequenza, allora si attacca nel seguente modo:
\begin{itemize}
    \item con la carta in coda al colore se si possiede almeno un onore:

        \cards{\textbf{K\Ten54\underline{3}}}\qquad
        \cards{\textbf{AJ95\underline{2}}}\qquad
        \cards{\textbf{\Ten865\underline{4}}}

    \item con la seconda carta dall'alto se non lo si possiede:

        \cards{\textbf{8\underline{6}43}}\qquad
        \cards{\textbf{9\underline{7}65}}
\end{itemize}
\end{regola}

\paragraph{Il Terzo di mano} I difensori hanno due riferimenti per immaginare dove siano gli onori che non vedono: la
logica del gioco e il `linguaggio delle carte' cui entrambi si attengono.

L'obiettivo del terzo di mano è cercare di vincere la presa ed, eventualmente, sacrificare i propri onori per affrancare
quelli del compagno. 

\begin{wraptable}[6]{l}[0cm]{6cm}

    \vspace{-.5cm}

    \hspace{1cm}
    \onesuitAll{82}{AT3}{Q96}{KJ754}

    \bigskip

    \hspace{1.375cm}
    \onesuitAll{J72}{A8}{QT64}{K953}
\end{wraptable}
\bigskip

Ovest attacca con il 4. Est deve giocare la Dama: se non lo facesse, il Giocante vincerebbe con il \Ten e farebbe 2
prese.
\vspace{.7cm}

Ovest attacca con il 3, Est al suo turno giocherà la Dama se Nord ha impegnato il J, il \Ten se Nord ha giocato la
cartina.
\medskip

\`E, inoltre, fondamentale, che il Terzo di mano eviti di bloccare il colore di attacco. Di conseguenza, nel caso
possieda onori corti (accompagnati da una sola carta) dovrà giocarli al primo giro.

Dovrà, infine, giocare il seme d'attacco ad ogni occasione, a meno che il la figura che vede al Morto sia tale da
consigliare un cambio di colore.

Una regola importante per il Terzo di mano:

\begin{regola}{Regola generale del Terzo di Mano}\label{regola:terzo di mano}
    \textsc{Quando ha carte equivalenti, il Terzo di mano gioca la più bassa della sequenza}
\end{regola}

Fa quindi il contrario del difensore che muove per primo un colore invece che giocare la \emph{testa}, egli gioca la \emph{coda}.

\medskip
\begin{tabular}{l l }
    
    \cards{\textbf{KQJ}}:&se muovete per primi il Re, se terzi il Fante\\

    \cards{\textbf{AKQ}}:&se muovete per primi il Asso, se terzi la Dama\\

    \cards{\textbf{J\Ten9}}& se muovete per primi il Fante, se terzi il 9
\end{tabular}
\medskip

La \autoref{regola:terzo di mano} permette la seguente deduzione:

\begin{regola}{Deduzione dell'attaccante}
    \textsc{La carta giocata dal Terzo di mano, quando è impegnata per prendere, esclude il possesso della carta
    immediatamente inferiore}
\end{regola}

\newpage

\section*{Esercizi}

\begin{minipage}{.45\textwidth}
\onesuitNE{754}{KQT3}

Il compagno attacca con il 2, che carta giocate?\footnotetext{\rotatebox[origin=c]{180}{Risposta: K}}
\end{minipage}\hfill
\begin{minipage}{.45\textwidth}
    \onesuitNE{543}{AK2}

    Il compagno attacca con il 6, in che ordine giocate le vostre carte?
    \footnotetext{\rotatebox[origin=c]{180}{Risposta:K, A, 2}}
\end{minipage}\hfill

\bigskip
\hspace{-.7cm}
\begin{minipage}{.45\textwidth}
    \onesuitAll{842}{Q}{J}{K9653}

    Dopo la presa: \cards{\textbf{3}}, \cards{\textbf{2}}, \cards{\textbf{J}}, \cards{\textbf{Q}}, chi ha l'Asso? E il
    Dieci, invece?  \footnotetext{\rotatebox[origin=c]{180}{Risposta: Entrambi Sud}
    }
\end{minipage}\hfill\qquad
\begin{minipage}{.45\textwidth}
    \onesuitAll{96}{K}{T}{Q8753}

Chi ha l'Asso? E il Fante?\footnotetext{\rotatebox[origin=c]{180}{Risposta: Sud e Est}}
\end{minipage}\hfill
\bigskip

\hspace{-.7cm}
\begin{minipage}{.45\textwidth}
    \onesuitAll{872}{A}{Q}{J9654}

    Chi ha il K?
    \footnotetext{\rotatebox[origin=c]{180}{Risposta: Non si può sapere}}
\end{minipage}\hfill
    
\end{document}

