\documentclass[../corsofiori.tex]{subfiles}
\externaldocument[I-]{../lezione1/Lezione1}
\externaldocument[II-]{../lezione2/Lezione2}
\setcounter{regolactr}{7}

\begin{document}
\setdefaults{colors=4A}

\setcounter{chapter}{2}
\chapter{Affrancamenti di lunga e di posizione}


Oltre a quelli studiati nella \autoref{I-chap:lezione1}, ovvero quelli \emph{di forza}, vi sono altre tipologie di
affrancamenti: quelli \emph{di lunga} e \emph{di posizione}.

\section{L'affrancamento di lunga}

Quando si possiedono molte carte in un seme è possibile che, dopo aver giocato ripetutamente in quel seme, le carte
rimaste, anche se di basso rango, siano diventate vincenti, perché gli altri giocatori le avranno esaurite.

Gli affrancamenti di lunga dipendono da due fattori:
\begin{itemize}
\item La lunghezza massima che abbiamo in quel seme (quante carte ha la mano che ne ha di più)
\item La \emph{divisione} delle restanti carte
\end{itemize}

\begin{wraptable}[6]{l}[0cm]{3.5cm}
\onesuitAll{762}{AKQ43}{T85}{J9}
\end{wraptable}

Sud possiede 5 carte (+ 3 del morto) e 5 sono in mano ai difensori. Mentre incassa le 3 vincenti, Sud conta quelle che
vengono giocate dai difensori. In questo caso, gli avversari hanno 3 e 2 carte per ciascuno\sidefootnote{Si dice che i
\emph{resti} sono 3-2 (\texttt{tre due)}}. Ora, essendo le ultime carte rimaste, \cards{\textbf{43}} sono affrancati e, quindi,
vincenti.

\begin{attenzione}{Attenzione}
Per effettuare un affrancamento di lunga è necessario che vi sia almeno una situazione favorevole!
\end{attenzione}
\newpage

    \onesuit{\textbf{?}}{\textbf{62}}{\textbf{?}}{\textbf{AKQ4}}

Sarà possibile affrancare il 4? No, perché noi abbiamo 6 carte e gli avversari 7: anche con la divisione più favorevole
nelle mani avversarie (4-3), uno degli avversari supererà il 4 al quarto giro.

Con la seguente figura:

\onesuit{\textbf{?}}{\textbf{653}}{\textbf{?}}{\textbf{AKQ742}}

Siate ottimisti e giocate \cards{\textbf{AKQ}}! La sola distribuzione che vi impedisce di affrancare tre prese è la 4-0,
estremamente improbabile. Ricordatevi, infatti, la seguente regola:


\begin{regola}{Regola della distribuzione dei resti}
    Le distribuzioni dei resti più probabili sono quelle con disparità minima, seguite da quelle con divisioni pari (se
    le carte rimanenti sono pari) o disparità leggermente maggiore (se le carte rimanenti sono dispari).
\end{regola}

    Quindi:
    \begin{center}
    \begin{tabular}{lcc}
        \toprule
        & \multicolumn{2}{c}{Resti}\\
        \cmidrule(l){2-3}
        N°& 1$^a$ & 2$^a$ \\
        \midrule
        5 & 3-2 & 4-1\\
        6 & 4-2 & 3-3\\
        7 & 4-3 & 5-2\\
        8 & 5-3 & 4-4\\
    \end{tabular}


    \end{center}


I colori lunghi sono sempre sorgenti di prese, anche quanto molto poveri di onori:

\begin{wraptable}[6]{l}[0cm]{3cm}

    \onesuitAll*{T9762}{8543}{AQ}{KJ}

\end{wraptable}

La \emph{linea N-S}\sidefootnote{Nel bridge si indica con \emph{linea} una coppia}, giocando due volte il colore,
affrancherà ben tre prese, nonostante non possieda nemmeno un Onore!

Capite ora il perché della \autoref{II-regola:scelta dell'attacco}? L'attacco è un'occasione per tentare un affrancamento di lunga.
Ricordatevi, però, che anche queste vincenti necessitano di \emph{rientri} o \emph{collegamenti} per essere incassate.

\newpage

\section{Affrancamento di posizione}

L'affrancamento di posizione è un tipo di affrancamento che richiede, invece che una distribuzione dei resti favorevoli,
una posizione di Onori favorevole.

Tutte le volte che un onore al momento non è vincente, ma potrebbe fornire una presa, non giocatelo direttamente:

\begin{regola}{Per l'affrancamento di posizione}
 Posizionatevi dalla parte opposta e giocate \emph{\textbf{verso}} quell'onore, in modo da utilizzarlo in terza
posizione; se la carta superiore è in mano all'avversario che gioca per secondo, la manovra avrà successo e il vostro
onore vincerà la presa, se invece è in mano al quarto\ldots pazienza.
\end{regola}

L'esito è, quindi, dipendente dalla posizione degli onori degli avversari.

Vi sono due manovre di affrancamento di posizione:

\begin{itemize}
    \item L'impasse consiste nel giocare verso un onore protetto da una carta superiore:
\end{itemize}

\begin{wraptable}[2]{l}[-.9cm]{2cm}
    \vspace{-1.0cm}
\onesuitNS*{AQ}{43}
\end{wraptable}

        Per fare due prese giocate il 3 verso la Dama: vincerete tutte le volte che Ovest ha il Re.

\begin{itemize}
    \item L'expasse consiste nel giocare verso un onore non protetto da una carta superiore:
\end{itemize}

\begin{wraptable}[4]{l}[-.9cm]{2cm}
    \vspace{-1.0cm}
\onesuitNS*{Q65}{A43}
\end{wraptable}

Per fare due prese, giocate il 3 verso la Dama, nuovamente, vincerete tutte le volte che Ovest ha il Re. Non ha senso,
invece, partire di Dama: Est coprirà con il Re, se lo possiede, impedendovi di fare presa.

Se invece il Re è in mano Ovest, farà una presa
mangiando un onore, invece che delle cartine, e a voi non resterà nessuna presa affracata.

\section{Per i difensori}

Per difendersi dall'affrancamento di lunga è necessario possedere una \emph{tenuta}\sidefootnote{Detta anche
\emph{arresto}, \emph{retta} o \emph{fermo}}: una figura o una singola carta con cui un giocatore, da solo o in
collaborazione con le carte del compagno, si oppone ad un affrancamento:
\newpage

\onesuitAll*{87}{AKT65}{QJ}{9432}

Nonostante la Dama e il Fante di Est cadano sotto Asso e Re e il \Ten si affranchi, il 9 di Ovest è una tenuta perché
non permette l'affrancamento immediato del 6 e del 5.

Quindi, fondamentale:


\begin{regola}{Mantenere le tenute}
    \textsc{Se vedete un colore lungo al morto e pensate che il compagno abbia meno carte di voi, non scartatene nemmeno una}
\end{regola}

Non è, invece, sempre possibile difendersi dagli affrancamenti di posizione, ma vi è una regola di comportamento:


\begin{regola}{Onore su onore e piccola su piccola}
    \textsc{In seconda posizione, giocate piccola se l'avversario ha iniziato con una piccola, coprite invece il suo
        onore se ha iniziato con un onore}
\end{regola}


Questo sacrificio conviene perché si può affrancare una presa per la propria linea:

\begin{wraptable}[7]{l}[0cm]{5cm}
\onesuitAll*{AQ94}{J65}{T72}{K83}
\end{wraptable}

Se Sud inizia con il J, Ovest deve mettere il K, seguendo la regola. In questo modo, il \Ten di Est farà una presa. Se
non coprisse, allora Sud farebbe tutte le prese: dopo aver fatto presa con il J, giocherebbe piccola alla Q e poi
l'Asso, facendo cadere il K e affrancando di lunga il 4.



\clearpage
\section*{Esercizi}

\bigskip


\begin{minipage}[h][3cm][t]{.3\linewidth}
    \onesuitNS*{K85}{763}

    Come giocate questa\\ figura?
\end{minipage}
\hfill
\begin{minipage}[h][3cm][t]{.3\linewidth}
    \onesuitNS*{AKJ4}{73}

    Come giocate questa\\ figura?
\end{minipage}
\hfill
\begin{minipage}[h][3cm][t]{.3\linewidth}
    \onesuitNS*{863}{AQJ}

    Come giocate questa\\ figura?
\end{minipage}
\hfill
\bigskip
\\
\begin{minipage}[h][3cm][t]{.3\linewidth}
    \onesuitNS*{A653}{QJT9}

    Come giocate questa\\ figura?
\end{minipage}
\hfill
\begin{minipage}[h][3cm][t]{.3\linewidth}
    \onesuitNS*{AQ53}{JT94}

    Come giocate questa\\figura?
\end{minipage}
\hfill
\begin{minipage}[h][3cm][t]{.3\linewidth}
    \onesuitNS*{AJ53}{KT92}

    Come giocate questa\\figura?
\end{minipage}
\hfill


\end{document}
