
\documentclass[../corsofiori.tex]{subfiles}
\begin{document}
\chapter{Piano di gioco a Senza Atout}

Come abbiamo detto, avere un obiettivo di prese permette al giocante un'organizzazione strategica delle proprie mosse, al
fine di ottenere il numero di prese necessarie.
Spesso il numero di vincenti è lontano da quello da ottenere. \`E, quindi, necessario, lavorare al fine di ottenere
affrancare le prese mancanti.

Dobbiamo partire facendoci due domande:
\begin{itemize}
    \item Quante prese ho a disposizione?
    \item Quante ne devo trovare per mantenere l'impegno?
\end{itemize}

A questo punto, si tratta di scegliere il colore o i colori da cui reperire le prese mancanti.
L'insieme di tutte queste valutazioni si chiama \emph{piano di gioco}.

A volte vi è un solo \emph{colore di sviluppo}, ed è evidente:

\newgame
\boardnr{1}
\northhand{86}{K5}{KQT75}{872}
\southhand{AKQ3}{A72}{J32}{A65}
\leftupper{\boardtext*}%
{\dealertext\quad}{\vulnertext}
\rightupper[2ex]{\contract:
3\,{\small SA}}{\declarer \south}{}
\rightlower[2ex]{\lead: 2\He}{}{}

\showAll*

\paragraph{Piano di gioco} Sull'attacco, prima di decidere da che parte prendere, Sud conta le vincenti: 3 a \pic,
2 a \cu, 1 a \fio. Servono altre 3 prese per arrivare a 9. \`E evidente che le quadri, da sole, possono fornirle.

Può l'avversario ostacolare la manovra di affrancamento? Sì, se rifiuta di prendere a quadri due volte. A quel punto ci
servirà una carta per trasferisci nella mano di Nord per incassare le \qu. Quindi è di fondamentale importanza conservare
il K\He al fine di prendere in Nord al momento opportuno. Sull'attacco staremo quindi bassi per prendere con l'Asso.
L'utilità delle carte vincenti non è solo quella di aggiudicarsi una presa, ma anche quella di permettere dei
trasferimenti tra Mano e Morto.



\medskip
\begin{regola}{Rientri}
    Il rientro è una carta che consente di far vincere la presa alla mano giusta nel momento giusto. Quindi,
    salvaguardate gli ingressi accanto alla lunga che volete affrancare.
\end{regola}
\medskip


Come vi ricorderete, il gioco a Senza Atout è una corsa all'affrancamento, e non siete gli unici a correrla: dovete
tenere conto anche degli avversari! \`E inutile scegliere una strada che vi farà fare le vostre prese solo dopo che
i difensori ne avranno incassate abbastanza per farvi fallire l'obiettivo.

\newgame
\boardnr{2}
\northhand{653}{32}{AQJT9}{764}
\southhand{AKQ}{A4}{872}{KQJT5}
\leftupper{\boardtext*}%
{\dealertext\quad}{\vulnertext}
\rightupper[2ex]{\contract:
3\,{\small SA}}{\declarer \south}{}
\rightlower[2ex]{\lead: K\He}{}{}

\showAll*

\paragraph{Piano di gioco} Sud conta 3 vincenti a \pic, 1 a \cu e 1 a \qu. Vi è la necessità di trovarne altre 4. Sia le
\fio che le \qu possono fornirle: le prime, attraverso un affrancamento di forza, le seconde sperando che il K\Di sia in
Ovest, accompagnato da non più di 3 cartine, ripetendo tre volte l'impasse.
Provando a pensare alle carte di Est-Ovest, ci rendiamo conto che la divisione per noi più favorevole delle \cu è 5-4
e che ora gli avversari hanno 4 vincenti di \cu da incassare. Non possiamo permetterci, quindi, di giocare \fio, perché
gli avversari farebbero 5 prese prima delle nostre 9. Bisogna, quindi, tentare l'impasse al Re di \qu per mantenere il
contratto.

Quando avete la possibilità di vincere l'attacco in Mano o al Morto, valutate prima cosa intendete fare subito dopo: vi
aiuterà a prendere dalla parte giusta:

\newgame
\boardnr{3}
\northhand{653}{864}{A653}{QJ3}
\southhand{AKQ}{A32}{K8}{AT985}
\leftupper{\boardtext*}%
{\dealertext\quad}{\vulnertext}
\rightupper[2ex]{\contract:
3\,{\small SA}}{\declarer \south}{}
\rightlower[2ex]{\lead: 2\Di}{}{}

\showAll*

\paragraph{Piano di gioco} Sud contra 3 vincenti a \pic, 1 a \cu, 2 a \qu e 1 a \fio. Deve trovarne altre 2. Le \fio
possono offrire 3 o 4 prese, a seconda della posizione del K\Cl. Da che parte ci conviene iniziare a giocarle? Da Nord,
perché questo ci permette di sottomettere l'eventuale Re di Est sotto l'asso (si gioca la Dama e se il Re non compare si
gioca una cartina in Sud, si prosegue con il Fante ripetendo lo stesso ragionamento e, infine, si gioca il 3 per il 9 di
Sud). Sapendo che la sua prossima mossa sarà giocare \fio dal Morto, il Giocante sa che è sua convenienza vincere
l'attacco con l'Asso del Morto.

\section{Mantenere i collegamenti}

Quando il colore che fornisce le prese è in una mano priva di ingressi, il Giocante deve cercare di affrancare il colore
e conservare la possibilità di incassarlo. Si deve, quindi, affrancare il colore mantenendo un \emph{collegamento}, ovvero un
rientro interno ad un colore affrancato.

\newgame
\boardnr{4}
\northhand{AK6}{A74}{QJ54}{A72}
\southhand{985}{T32}{63}{K8653}
\leftupper{\boardtext*}%
{\dealertext\quad}{\vulnertext}
\rightupper[2ex]{\contract:
1\,{\small SA}}{\declarer \south}{}
\rightlower[2ex]{\lead: 4\Sp}{}{}

\showAll*

\paragraph{Piano di gioco} Solo 5 vincenti, ma le \fio potrebbero fornire 2 prese di lunga, ammesso che il colore sia
diviso 3-2. Fare tutte e 5 le prese è impossibile, una presa spetta necessariamente ad Est-Ovest. Stabilito questo, Sud
affronterà subito le \fio giocando:
\renewcommand{\itemautorefname}{Caso}
\begin{enumerate}[label=\alph*)]
    \item\label{casoa} Giocare Asso, Re e \fio a "dare"
    \item\label{casob} Giocare il 2\Cl per il 3\Cl, cedendo la fiori immediatamente
\end{enumerate}
Ammesso che il colore sia ben diviso, la differenza è notevole: nel \autoref{casoa} il giocante ha due \fio affrancate,
ma nessun modo per incassarle. Nel \autoref{casob}, appena potrà entrare in presa, giocherà l'A\Cl, poi fiori per il
K e tutte le fiori a seguire: 1\SA mantenuto!

Avete appena assistito ad una manovra molto frequente: il \emph{Colpo in bianco}\ldots

\medskip
\begin{regola}{Colpo in bianco}
    Il colpo in bianco è la cessione immediata di una presa (che comunque si sarebbe ceduta dopo) allo scopo di
    mantenere le comunicazioni nel colore.
\end{regola}


\section{Colori comunicanti e colori bloccati}

Si dice che un colore comunica quanto è possibile trasferire la presa da una mano all'altra, bloccato quando questo non
accade:

\smallskip
\begin{center}
\onesuitNS{KQ72}{AJ83}\qquad\onesuitNS{QJT9}{AK}\hfill
\end{center}
\smallskip

La figura a sinistra consente l'incasso fluido di 4 prese, quella di destra no. Questa impossibilità è dovuta al fatto
che il lato corto contiene tutte le carte più alte del lato lungo. Sappiamo già che per evitare situazioni di blocco
è bene partire giocando gli onori dal lato corto.

\newgame
\boardnr{5}
\northhand{972}{A7}{JT952}{J32}
\southhand{AK3}{K85}{KQ3}{A854}
\leftupper{\boardtext*}%
{\dealertext\quad}{\vulnertext}
\rightupper[2ex]{\contract:
3\,{\small SA}}{\declarer \south}{}
\rightlower[2ex]{\lead: 2\He}{}{}

\showAll*

\paragraph{Piano di gioco} Sud conta le vincenti: 2 a \pic, 2 a \cu, 1 a \fio. La sorgente di prese e \qu, che può
fornire le 4 prese mancanti. Sud deve pianificare attentamente il giocando:
\begin{enumerate}
    \item Vincere l'attacco con il Re, in modo da lasciare l'Asso al morto come rientro, accanto al colore che vuole
        affrancare
    \item Giocare \qu partendo evitando di bloccare il colore: la carta con cui partire è un onore. Se l'avversario si
        rifiuta di prendere, giocherà l'altro onore e se l'Asso non è ancora uscito, giocherà la cartina catturata dal
        Fante del morto.
\end{enumerate}


A volte un blocco non è irrimediabile, purché dalla parte lunga esista, affianco alle prese da incassare, un rientro.


\newgame
\boardnr{6}
\northhand{6542}{T3}{KQJT2}{54}
\southhand{A73}{AKQJ2}{A}{AK32}
\leftupper{\boardtext*}%
{\dealertext\quad}{\vulnertext}
\rightupper[2ex]{\contract:
7\,{\small SA}}{\declarer \south}{}
\rightlower[2ex]{\lead: K\Sp}{}{}

\showAll*


\paragraph{Piano di gioco} Il Giocante ha a disposizione 13 vincenti: 1 a \pic, 5 a \cu, 5 a \qu e 2 a \fio.
Il colore di \qu è, però, bloccato, ma le carte del morto sono in grado di vincere una presa grazie al \Ten\He.
Questa carta, oltre a essere vincente, consentirà al giocante di incassare tutte le vincenti di \qu, dopo aver incassato
(sbloccato) l'Asso di \qu.

\medskip
\begin{regola}{Rimediare ai blocchi}
    Se vi accorgete di avere un colore bloccato, fate in modo che gli altri colori lavorino al fine di superare il
    blocco.
\end{regola}


\end{document}
