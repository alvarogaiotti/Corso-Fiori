\documentclass[../corsofiori.tex]{subfiles}
\setcounter{regolactr}{14}
\setcounter{chapter}{4}
\begin{document}
\chapter{Il gioco con Atout}
Il fatto che nel gioco ad Atout si realizzino prese non solo con le carte alte ma anche con le cartine d'Atout lo rende
preferibile rispetto al gioco a Senza, a condizione che la coppia abbia \emph{un numero di carte pari o superiore a 8 in
un seme}. Quando due compagni trovano l'incontro di 8 o più carte in un colore si dice che hanno trovato un
\textsc{fit}.
\
Se vi è un fit conveniente (\Sp o \He\sidefootnote{Questi si dicono colori \emph{maggiori}.}: 30 punti ogni presa) si
gioca con questa atout. Se non vi è fit in questi due semi, allora si gioca a \SA (30 punti ogni presa, 40 la prima). Se
non è possibile giocare a \SA, si gioca a \Di o \Cl\sidefootnote{Questi colori si dicono \emph{minori}.} (20 punti ogni presa).

\section{Differenze di valutazione della mano nel gioco a colore}

Il fatto di essere scelto come Atout rappresenta una rivalutazione per un colore, a maggior ragione se è composto da
carte deboli.

\hand*!{543}{QJT987}{654}{2}

Questa mano, nel gioco a Senza, non vale nulla. Ma se \cu fosse atout, il possessore di queste carte avrebbe la certezza
assoluta di fare almeno 4 prese.
\begin{regola}{Affrancabili del seme di Atout}
    \textsc{Il fatto di elevare al ruolo di atout un colore fa sì che le sue carte affrancabili siano da considerarsi
    come vincenti}
\end{regola}
Ma il potere di taglio spetta anche agli avversari, quindi:

\begin{regola}
    \textsc{Le vincenti nei colori a lato delle atout sono vincenti relative: il Giocante le potrà liberamente incassare
    \textbf{solo quando} avrà eliminato le atout in mano agli avversari. Questa operazione è detta battere le atout.}
\end{regola}

\newgame
\boardnr{1}
\northhand{874}{T9}{KQJ2}{AK63}
\southhand{3}{AKQJ75}{A54}{987}
\westhand{AKQ92}{432}{63}{T52}
\easthand{JT65}{86}{T987}{QJ4}
\leftupper{\boardtext*}%
{\dealertext\quad}{\vulnertext}
\rightupper[2ex]{\contract:
4\,\He}{\declarer \south}{}
\rightlower[2ex]{}{}{\lead: A e K\Sp}

\showAll*

Sud taglia con una cuori la continuazione di K\Sp; ora, se provasse subito ad incassare le \qu vincenti, Ovest
taglierebbe al terzo giro. Quindi, per prima cosa, Sud dovrà battere le atout, ovvero incassare 3 giri di \cu (quanto
basta per togliere le atout agli avversari), e solo dopo incassare le vincenti senza timore che vengano tagliate.
Arriverà così a 12 prese: 6\He, 4\Di, 2\Cl. Giocando a Senza avrebbe perso le prime 5.


\section{Il potere di Controllo}
Quando il gioco è ad atout, non c’è più da temere per i colori corti: il taglio può interrompere le vincenti
dell’avversario. Il taglio può essere effettuato sia dalla mano che dal morto; ovviamente la prima mano a poterlo
esercitare sarà quella che per prima esaurisce le carte nel colore giocato. Nell’esempio sopra Sud ha esercitato il
Potere di controllo quando ha tagliato con il 5\He il Re di picche di Ovest.
Esercitare questo taglio non gli ha dato prese supplementari; erano in previsione 6 prese a \cu e 6 resteranno (5
incassate + 1 taglio = 6).

\section{Il potere di Allungamento}
Nel gioco ad Atout è possibile, a volte, ottenere dal colore scelto come atout più prese di quante ne darebbe se si
giocasse a Senza.

\newgame
\boardnr{2}
\northhand{T95}{4}{A762}{AT842}
\southhand{AKQJ73}{A63}{K9}{65}
\easthand{864}{QT952}{Q83}{93}
\westhand{2}{KJ87}{JT54}{KQJ7}
\leftupper{\boardtext*}%
{\dealertext\quad}{\vulnertext}
\rightupper[2ex]{\contract:
4\,\Sp}{\declarer \south}{}
\rightlower[2ex]{}{}{\lead: K\Cl}

\showAll*

\paragraph{Piano di gioco} Vincenti: 6 a \pic, 1 a \cu, 2 a \qu e 1 a \fio.
Ma si può arrivare a 12 se si ricavano 8 prese dalle \pic invece che 6.
Sud vince con l'Asso, gioca subito Cuori per l'Asso e taglia una \cu. Ora gioca \pic per un onore e ancora \cu, tagliata
da una \pic. Ora rientra in mano con il K\Di, completa la battuta delle atout e incassa l'A\Di. Ha realizzato otto prese
a \pic: 6 in mano più due tagli fatti al Morto. Questo si dice \emph{allungare le atout}. Se Sud avesse, come prima
cosa, eliminato le atout avversarie, non avrebbe potuto ottenere 12 prese, perché sarebbero sparite le atout dalla parte
corta. Il potere di allungamento delle atout è dato dalla possibilità che esse offrono, tagliando, di far aumentare le prese
nel colore di atout.

Ovviamente, un giocatore realizza un allungamento delle prese in atout solo ed esclusivamente qualora il taglio aumenti
almeno di una le prese che normalmente avrebbe fatto nel colore.

\section{Potere di Affrancamento}

\newgame
\boardnr{3}
\westhand{74}{QT53}{KQT84}{Q6}
\northhand{KJ3}{A2}{65}{A98753}
\southhand{AQ8652}{986}{A3}{K2}
\easthand{T9}{KJ74}{J972}{JT4}
\leftupper{\boardtext*}%
{\dealertext\quad}{\vulnertext}
\rightupper[2ex]{\contract:
4\,\Sp}{\declarer \south}{}
\rightlower[2ex]{}{}{\lead: K\Di}

\showAll*

\paragraph{Piano di gioco} Vincenti: 6\Sp, 1\He, 1\Di, 2\Cl. Affrancabili: 3 prese di lunga a \fio se i resti sono 3-2.

Sud vince l'attacco, batte le atout in due giri, incassa Re e Asso di \fio e taglia in mano una \fio e cadono tutte. Ora
le tre \fio del Morto sono franche e raggiungibili con il K\He. 13 prese fatte.

Il potere di affrancamento consente, tagliando, di affrancare le cartine in un colore lungo senza cedere prese
all'avversario.

\section{Consigli per i difensori}

Quando si gioca ad Atout non ha più molto senso cercare di affrancare il proprio colore lungo, perché anche vi si
riuscisse, il giocante taglierà le nostre vincenti. A volte sarà meglio cercare di affrancare prese rapide, o sperare
nei tagli.
\Sp KQ7 \He62 \Di J9732 \Cl A72, il contratto è “4\He”: K\Sp è meglio del 2\Di.
\Sp2 \He A62 \Di J982 \Cl Q8752, il contratto è “4\He”: un attacco possibile è il 2\Sp.

\begin{regola}{Attacco da sequenza nei contratti ad Atout}
    Se il contratto avversario è in atout, per l'attacco da sequenza bastano due onori contigui
\end{regola}

Gli attacchi da colori molto corti sono consigliati, in quanto hanno la ragionevole motivazione di provare a tagliare.

\begin{regola}{Attacco dalla corta}
    Quando si attacca da due carte, si sceglie sempre la più alta, che sia una cartina od un onore:
    \begin{itemize}
        \item Con 74: il 7
        \item Con \Ten2: il \Ten
        \item Con A9, l'Asso
    \end{itemize}
\end{regola}

I difensori possono concertare ottimi controgiochi, confidando nelle deduzioni che derivano dagli accordi! Mettetevi nei
panni di Est:

\newgame
\boardnr{3}
\northhand{AK65}{J74}{T864}{32}
\easthand{872}{A52}{A932}{654}
\leftupper{\boardtext*}%
{\dealertext\quad}{\vulnertext}
\rightupper[2ex]{\contract:
4\,\He}{\declarer \south}{}
\rightlower[2ex]{}{}{\lead: J\Di}

\showAll*

Est sa che Ovest ha per certo J e una cartina, oppure J secco, perché? Perché vede il \Ten\Di al morto, che esclude che
si tratti di un attacco da sequenza! Quindi prenderete con il vostro Asso e rigiocherete \qu. Se Ovest non taglia a questo
giro, quando fermerete la battuta delle atout con l'A\He, giocherete nuovamente \qu, garantendo il taglio al compagno!

Un attacco è assolutamente da evitare, quando il contratto è ad Atout.

\begin{regola}{Attacco sotto asso}
    Non si deve, MAI, attaccare \emph{sotto asso}, ovvero attaccare in un colore nel quale si possiede l'Asso.
    Il grosso rischio è, infatti, regalare una presa che non tornerà indietro.

    \onesuitAll*{9732}{K}{Q85}{AJT64}

    Se Ovest attacca con il J\Cl, come farebbe a Senza Atout, permetterà a Sud di realizzare il Re secco, e non potrà
    più fare presa con l'A\Cl, in quanto questo verrà tagliato!


\end{regola}

\end{document}
