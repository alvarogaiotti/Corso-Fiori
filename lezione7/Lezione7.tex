\documentclass[../corsofiori.tex]{subfiles}
\setcounter{regolactr}{21}
\setcounter{chapter}{6}
\begin{document}

\chapter{I contratti e la dichiarazione di apertura}
\setdefaults{colors=4A}

\paragraph*{Dichiarare parziali, manche e slam: i requisiti di coppia}

Le carte che statisticamente realizzano più frequentemente le prese sono gli assi e le tre figure di ogni colore. Un
giocatore americano, Milton Work, ha attribuito a queste carte un valore numerico per avere un \textsc{riferimento}
(esclusivamente statistico) circa le possibilità per ogni linea di realizzare una o più prese:
\begin{center}
\cards{A} = 4 punti\qquad
\cards{K} = 3 punti\qquad
\cards{Q} = 2 punti\qquad
\cards{J} = 1 punto
\end{center}

Quindi in ogni seme ci sono 10 punti (A+K+Q+J) e un totale di 40 punti nel
mazzo. Quando un giocatore afferma di avere "14 punti", di lui si saprà che può ad esempio avere tre assi + una dama,
oppure un asso + tre re + un fante, oppure due assi + due re o altro ancora, per un totale di 14. Esempio:

\hand{AQ854}{K9}{J87}{AKJ}: 18 punti

\hand{7}{AKQ64}{A32}{T953}: 13 punti


Statisticamente è verosimile che la coppia che possiede più punti sia in grado di realizzare più prese dell'altra;
questo non sempre è vero, ma poiché si tratta di un punto di riferimento statisticamente valido, potremo ritenere
giustificati da parte di una coppia i seguenti contratti:

\begin{table}[h]
   \centering
   \begin{tabular}{llc}
       \toprule
    Giustificativo & Denominazione contratto & Contratto\\
    \midrule
    con 20 - 24 punti coppia&contratti parziali& tutti, tranne i seguenti\\
con 25 - 31&contratti di Manche            &3NT, 4\Sp, 4\He, 5\Di, 5\Cl \\
con 32 - 36&contratti di piccolo Slam      &6NT, 6\Sp, 6\He, 6\Di, 6\Cl \\
con 37 o più&contratti di grande Slam      &7NT, 7\Sp, 7\He, 7\Di, 7\Cl\\
\bottomrule


\end{tabular}
\end{table}





\paragraph*{Ordine prioritario dei tipi di contratto}
I contratti a Cuori e Picche sono la prima scelta; il vantaggio dei tagli consente di
fare molte prese anche se la forza onori è modesta. In alternativa, qualora non ci sia
fit di 8 carte a \He o \Sp, i contratti più convenienti sono quelli a Senza. I contratti a Fiori
e Quadri rendono meno, quindi sono la terza scelta.

\section*{L’Apertura}
L’apertura è la prima dichiarazione diversa da Passo fatta al tavolo. Aprire è una
libera scelta, non un obbligo: è anche possibile che nessuno apra (“tutti passano”),
se nessuno ha i requisiti per farlo. C’è un’unica “apertura” in ogni smazzata. Il
sistema dichiarativo che impareremo si chiama ``Quinta nobile, Quadri quarto''.
La decisione di Aprire o Passare dipende dalla Forza della mano:
\begin{regola}{Regola per l'apertura}
    Si apre quando si hanno almeno 12 punti onori
\end{regola}

\begin{tabular}{cl}
    \hand{AQJxx}{KQJx}{x}{Txx} & 13 punti: si apre\\
    \hand{xx}{xx}{KQJxx}{KJxx} & 10 punti: si passa
\end{tabular}
\smallskip

La forza della mano determina anche il Livello a cui aprire:

\begin{table}[htpb]
    \centering
    \begin{tabular}{cl}
        \toprule
        Punteggio&Livello di apertura\\
        \midrule
        0--11 & \textsc{zero} (Passo)\\
        12--20 & \textsc{uno} (1\Cl, 1\Di, 1\He, 1\Sp, 1NT)\\
21+ & \textsc{due} (2\Cl, 2\Di, 2\He, 2\Sp, 2NT)\\
\bottomrule
    \end{tabular}
\end{table}

Il tipo di apertura è determinato dalla distribuzione.  La distribuzione di una mano è l’insieme delle lunghezze dei
quattro semi Quando si dice, ad esempio, di avere “una 5-4-3-1” si intende una mano che contiene un colore quinto, uno
quarto, uno terzo e un singolo. Ci sono due principali famiglie:

Le mani bilanciate. Le distribuzioni bilanciate sono queste tre:
\begin{enumerate}[label=-]
\item la 4333
    (es : \hand{xxx}  {Qxx}  {Kxx} {AKxx})\sidefootnote{Quando si scrivono delle mani è uso utilizzare le x per
    rappresentare cartine poco rilevanti}
\item la 4432
    (es :\hand   {KJxx}   {AQx}   {QJxx}   {AK})
\item la 5332
    (es:\hand   {Axx}   {xx}   {KJxxx}   {KQx})
\end{enumerate}

Le mani sbilanciate sono tutte le altre.

\subsection{Come aprire con mani bilanciate (12-20)}
Con tutte le mani da 15 a 17 punti: 1NT


\begin{enumerate}[label={}]
\item \hand{KQ76}{AQ62}{KQ7}{98} $\rightarrow$ 1NT

\item \hand{J6} {KQ7} {KJ2} {AJ854} $\rightarrow$ 1NT
\end{enumerate}

\begin{regola}{Deduzione sull'apertura}
    Quando il compagno apre di 1 a colore non ha mai una mano bilanciata
di 15-17 (avrebbe aperto con 1NT)
\end{regola}

Con le mani che hanno di meno (12-14 punti) o di più (18 -20) si apre di:
\begin{enumerate}
    \item 1\Sp, se ci sono 5 carte a \Sp, altrimenti\ldots
    \item 1\He, se ci sono 5 carte a \He, altrimenti\ldots
    \item 1\Di, se ci sono almeno 4 carte, altrimenti\ldots
    \item 1\Cl, anche con sole 2 carte.
\end{enumerate}

\begin{enumerate}[label={}]
\item \hand{K943}{A2}{Q97}{KJ52} $\rightarrow$ 1\Cl

\item \hand{43} {AQJ82} {Q97} {A72} $\rightarrow$ 1\He

\item \hand{AQJ3} {KJ62} {Q7} {853} $\rightarrow$ 1\Cl

\item \hand{AKJ8} {Q982} {K8} {AQ3} $\rightarrow$ 1\Cl

\item \hand{QJ7}{A954}{AQ86} {J5} $\rightarrow$ 1\Di
\end{enumerate}


\subsection{
Come aprire con le mani sbilanciate (12-20)
}
Con le monocolori si apre di “uno” nell’unico colore lungo:

\begin{enumerate}[label={}]
\item \hand{53} {AJ4} {K9} {AJT762} $\rightarrow$ 1\Cl

\item \hand{6} {AQ9853} {KQ7} {K72} $\rightarrow$ 1\He
\end{enumerate}

Con un colore quarto e uno più lungo si apre di “uno” nel lungo:

\begin{enumerate}[label={}]
\item \hand{9} {QJ54}{AQ764}{A97}$\rightarrow$ 1\Di

\item \hand{J2} {A4}{KJ87} {KQ764}$\rightarrow$ 1\Cl
\end{enumerate}




Con due colori entrambi lunghi (quinti o più) si apre nel più alto di rango:

\begin{enumerate}[label={}]
\item \hand{K2} {4} {KJ876} {AQ873} $\rightarrow$ 1\Di



\item \hand{KQ985} {-} {AJ6432} {Q3} $\rightarrow$ 1\Sp

\end{enumerate}
Con tre colori quarti si apre di 1\Di se ci sono 4 carte, altrimenti 1\Cl

\begin{enumerate}[label={}]
\item \hand{J} {AKJ5} {K986} {AQJ4} $\rightarrow$ 1\Di

\item \hand{AJ85} {KQ65} {2} {KQJ4} $\rightarrow$ 1\Cl
\end{enumerate}

Quando “aprite” offrite garanzie di un certo punteggio, comunque distribuito, non promettete affatto di avere onori nel
colore che avete annunciato. Non lasciatevi
mai influenzare da dove siano dislocati gli onori:
\begin{regola}{Regola sul colore di apertura}
    
La scelta del colore di apertura dipende dalle lunghezze
\end{regola}

L’apertura di 1\Cl è di lunghezza ambigua: il colore può essere presente davvero, o in caso estremo può presentare due
sole carte (unico caso: 4\Sp4\He3\Di2\Cl) 
\subsection{
Le aperture a livello 2
}
Mostrano mani troppo forti per essere aperte a livello uno:
punti onori 21 +, oppure una mano che suggerisce 8 ½ - 9 vincenti

Se sono bilanciate, da 21 a 22 punti: 2NT

Se sono sbilanciate si apre nel colore lungo, secondo gli stessi schemi visti per l’apertura a livello 1, tranne:
    
\begin{enumerate}[label={}]

\item \hand{AQJ975} {AKJ} {5} {KQ4} $\rightarrow$ 2\Sp

\item \hand{AKT986} {AKJT92} {A} {-}$\rightarrow$ 2\Sp

\item \hand {5} {AK95} {AKQ98} {AQ9}$\rightarrow$ 2\Di

\item \hand {KQJ4} {AK} {3} {AKQJ76}$\rightarrow$ 2\Cl

\item \hand {KQ} {AKJ9} {KQ97} {AK5}$\rightarrow$ 2\Cl 

\end{enumerate}
Il colore di apertura è sempre di almeno 6 carte, può essere di 5 solo se ha un altro
colore almeno quarto a fianco. Quindi mai una 5332; con tali distribuzioni è
conveniente trattare la mano come una bilanciata di equivalente forza:
\begin{enumerate}[label={}]
\item \hand{AQ9} {AKJ} {AJ} {K9754} $\rightarrow$2NT

\item \hand {AJ4} {K97} {AK962} {AQ} $\rightarrow$2NT

\item \hand {KQ5} {K95} {AKQ85} {A9} $\rightarrow$2NT

\end{enumerate}
Mentre l’apertura di 2NT è
precisamente limitata (21- 22), le
aperture di 2\Cl, 2\Di, 2\He, 2\Sp non
hanno un limite superiore.
Chiedono al compagno di avere
un’altra possibilità di dichiarare,
quindi:

\begin{regola}{Regola sulle aperture a livello 2}
Sulle aperture di 2 a colore è vietato dire passo
\end{regola}

L'apertura di 2\Cl è ambigua fino al giro successivo: in pratica essa è un contenitore in cui alloggiano due tipi di
mano, la bilanciata fortissima (23 e oltre) e la mano a base fiori. L'ambiguità verrà risolta al giro successivo, in
quanto l'Apertore dirà:
\begin{enumerate}[label={-}]
\item i senz'atout, con il tipo di mano bilanciato

\item un nuovo colore o ancora le fiori con la bicolore fiori - \Sp/\He/\Di , o la monocolore di fiori
\end{enumerate}

\begin{center}
\begin{biddingpair}
    2C & 2D\\
    2S & \\
\end{biddingpair}\qquad
\begin{biddingpair}
    2C & 2S\\
    2N & \\
\end{biddingpair}
\end{center}

Quindi il possesso reale delle fiori come palo lungo di
base viene confermato nel momento in cui l'Apertore
non replica a senz'atout: nella prima sequenza
l'Apertore ha 4\Sp e 5 o più \Cl. Nella seconda sequenza
ha la bilanciata forte.
\end{document}
