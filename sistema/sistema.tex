\PassOptionsToPackage{dvipsnames,x11names,svgnames}{xcolor}
\PassOptionsToPackage{colors=4A}{onedown}
\PassOptionsToPackage{usenames,dvipsnames,pdftex}{color}
\documentclass[a4paper,italian,12pt]{article}
\usepackage[margin=2cm]{geometry}
\usepackage{lscape}
\usepackage[T1]{fontenc}
\usepackage[utf8]{inputenc}
\usepackage{newcent}
\usepackage{helvet}
\usepackage{graphicx}
%\usepackage{fancyhdr}
%\pagestyle{fancy}
\usepackage{wrapfig}
\usepackage[printonlyused]{acronym}
\newacro{ldg}[LDG] {Livello di guardia}
%%%%%%%%%%%%%%%
%  Pagestyle  %
%%%%%%%%%%%%%%%
\usepackage[colors=4A]{onedown}
\usepackage[italian]{babel}
\usepackage{enumitem}
\usepackage[most]{tcolorbox}
\usepackage{multicol}
\usepackage{xr}
%%% Colors
\definecolor{linkcolor}{HTML}{E23812}
\definecolor{azzurro}{HTML}{1ECBE1}
\definecolor{titlepagecolor}{HTML}{FF0800}
\definecolor{MidnightBlue}{HTML}{191970}
\definecolor{arancionediam}{HTML}{FFA500}
\definecolor{OLIVEGREEN}{rgb}{0.33, 0.42, 0.18}
\definecolor{MIDNIGHTBLUE}{HTML}{191970}
\definecolor{REDORANGE}{HTML}{FF5349}
\definecolor{RED2}{HTML}{EE0000}
%\definecolor{titlepagecolor}{RGB}{255,40,0}
%{cmyk}{1,.60,0,.40}

\usepackage[pdftex, pdfborder={0 0 0}, colorlinks=true, linkcolor=linkcolor]{hyperref}
\frenchspacing

\usepackage{txfonts} % For \varheartsuit and \vardiamondsuit
%\usepackage[usenames,dvipsnames,monochrome]{color} % dvipsnames necessary to made PDFLaTeX work.

\usepackage{listliketab}
\usepackage{latexsym} % \Box
\usepackage{pbox} % \Box
\usepackage{parskip} % line between paragraphs
\usepackage{pgfornament}
\usepackage{booktabs}
\usepackage{tikz}
\usetikzlibrary{shapes.geometric,calc,arrows.meta}
% suits


\newcommand{\BC}{\textcolor{OliveGreen}{$\clubsuit$}}
\newcommand{\BD}{\textcolor{RedOrange}{$\vardiamondsuit$}}
\newcommand{\BH}{\textcolor{Red2}{$\varheartsuit${}}}
\newcommand{\BS}{\textcolor{MidnightBlue}{$\spadesuit${}}}

%suits for pdf-friendly titles
\newcommand{\pdfc}{\texorpdfstring{\BC{}}{C}}
\newcommand{\pdfd}{\texorpdfstring{\BD{}}{D}}
\newcommand{\pdfh}{\texorpdfstring{\BH{}}{H}}
\newcommand{\pdfs}{\texorpdfstring{\BS{}}{S}}

\newcommand*\numcircledtikz[1]{\tikz[baseline=(char.base)]{
\node[shape=circle,draw,inner sep=1.2pt] (char) {#1};}} 

%Titlepage
\addto\captionsitalian{
    \renewcommand{\chaptername}{Lezione}
    \renewcommand*{\chapterautorefname}{Lezione}%
}
\addto\extrasitalian{
    \renewcommand*{\chapterautorefname}{Lezione}%
}
\newcommand\SA{{\smaller{SA}}\xspace}
\newcommand\pic{Picche\xspace}
\newcommand\cu{Cuori\xspace}
\newcommand\qu{Quadri\xspace}
\newcommand\fio{Fiori\xspace}

\newenvironment{bidtable}
{\begin{tabbing}

    xxxxxx\=xxxxxx\=xxxxxx\=xxxxxx\=xxxxxx\=xxxxxx\=xxxxxx\=xxxxxx\=xxxxxx\=xxxxxx\=\kill}
{\end{tabbing} }%

\newenvironment{sviluppi}
{\begin{tcolorbox}[colframe=azzurro,title=Sviluppi particolari]}
    {
\end{tcolorbox} }%

\newcounter{attenzionectr}
\newcounter{regolactr}
\newenvironment{attenzione}[1]
{
    \refstepcounter{attenzionectr}
\begin{tcolorbox}[colframe=red!80!white,title=#1]}
    {
\end{tcolorbox} }%
\newtcolorbox[auto counter, list inside=regole]{regola}[2][]{%
    colframe=green!40!black,
    title={Regola \thetcbcounter: #2}
}
\AtBeginEnvironment{regola}{\medskip}
\AtEndEnvironment{regola}{\medskip}
%\newenvironment{regola}[1]
%{
%    \refstepcounter{regolactr}
%    \medskip
%    \begin{tcolorbox}}
%    {
%\end{tcolorbox}
%    \medskip
% }%
\newcommand{\regolactrautorefname}{Regola}
% \newenvironment{bidding}%
% {\begin{tabbing}
% xxxxxx\=xxxxxx\=xxxxxx\=xxxxxx \kill
% }{\end{tabbing} }%end bidding

% writing hands
\newcommand{\cards}[1]{\textsf{#1}}
%\newcommand{\spades}[1]{\BS\cards{#1}}
%\newcommand{\hearts}[1]{\BH\cards{#1}}
%\newcommand{\diamonds}[1]{\BD\cards{#1}}
%\newcommand{\clubs}[1]{\BC\cards{#1}}
\newcommand{\void}{--}
\newcommand{\vhand}[4]{\spades{#1}\\\hearts{#2}\\\diamonds{#3}\\\clubs{#4}}

% The \Box should always appear the same distance from the left margin
\newcommand\onesuit[4]%
{%
    \begin{center}%
        \begin{tabular}{>{\hfill}p{3cm}cp{3cm}}
            & \cards{#2} \\%
            \cards{#1}& $\Box$    & \cards{#3} \\%
                      & \cards{#4} %
        \end{tabular}
    \end{center}%
}

% A special command if the south hand is not shown to avoid whitespace
\newcommand\onesuitenw[3]%
{%
    \begin{center}%
        \begin{tabular}{>{\hfill}p{3cm}cp{3cm}}%
                & \cards{#2} \\%
            \cards{#1}& $\Box$    & \cards{#3}%
        \end{tabular}%
    \end{center}%
}

\newcommand\dealdiagram[5]%
{%
    \begin{center}%
        \begin{tabular}{>{\hfill}p{3cm}cp{3cm}}
            \pbox{20cm}{\small #5}& \pbox{20cm}{#2} \\%
            \pbox{20cm}{#1}& $\Box$    & \pbox{20cm}{#3} \\%
                           & \pbox{20cm}{#4} %
        \end{tabular}
    \end{center}%
}

\newcommand\dealdiagramenw[4]%
{%
    \begin{center}%
        \begin{tabular}{>{\hfill}p{3cm}cp{3cm}}
            \pbox{20cm}{\small #4}& \pbox{20cm}{#2} \\%
            \pbox{20cm}{#1}& $\Box$    & \pbox{20cm}{#3} \\%
        \end{tabular}
    \end{center}%
}

\newcommand\dealdiagramew[2]%
{%
    \begin{center}%
        \begin{tabular}{>{\hfill}p{3cm}cp{3cm}}
            \pbox{20cm}{#1}& $\Box$    & \pbox{20cm}{#2} \\%
        \end{tabular}
    \end{center}%
}

%\newcommand*{\plogo}{\includegraphics[scale=.5, trim={1cm 1cm 1cm 1cm}, clip]{quote associative 2022.jpg}} % Generic dummy publisher logo

\title{Sistema Allievi}
\author{Circolo Bridge Pordenone \fbox{$\textcolor{red}{\mathcal{PN}}$}}

\begin{document}
%\definecolor{green}{named}{OliveGreen}
%\definecolor{orange}{named}{RedOrange}
%\definecolor{red}{named}{Red2}
%\definecolor{blue}{named}{MidnightBlue}
\setdefaults{colors=4A}
%\maketitle
\thispagestyle{empty}
\pagestyle{empty}
%----------------------------------------------------------------------------------------
%   TITLE PAGE
%----------------------------------------------------------------------------------------
%----------------------------------------------------------------------------------------

\begin{large}
    Quinta nobile, quadri quarte. 1\SA 15--17. 2\SA\ 21--22. No transfer. No surlicite.
\end{large}

\section*{Aperture}
\begin{large}
    \begin{bidtable}
        1\Cl\>\> 2+\Cl, 12--20 punti, può contenere bilanciata 18--20\\
        1\Di\>\> 4+\Di, 12--20 punti, può contenere bilanciata 18--20\\
        1\He\>\> 5+\He, 12--20 punti, può contenere bilanciata 18--20\\
        1\Sp\>\> 2+\Sp, 12--20 punti, può contenere bilanciata 18--20\\
        1\SA\>\> Bilanciata 15--17 punti, anche con 5\He/\Sp\\
        2\Cl\>\> 2+\Cl, 21+ punti, naturale sbilanciata o bilanciata 23+, forzante\\
        2\Di/\He/\Sp \> \> Naturali sbilanciati, 21+ punti, forzanti\\
        2\SA\>\> 21--22 bilanciati
    \end{bidtable}
\end{large}

\section*{Sviluppi}

\begin{bidtable}
    1\Cl/\Di--\+\\
    1\Di,\He,\Sp,\SA \> \> Naturali, 4+\Di/\He/\Sp, 5+ punti; 1\SA\ 5--10 senza 4e\\
    2\Cl \>\> Naturale forzante manche, 4+\Cl\\
    2\Di \>\> Su 1\Cl non previsto, 4+\Di 5--10 su 1\Di\\
    2\SA \>\> Invitante\\
    3\Cl \>\> Invitante, 6+\Cl\\
    3\Di \>\> Su 1\Di: invitante, 4+\Di\\
    3\SA \>\> 12--16 punti, bilanciato
\end{bidtable}

\paragraph{Legge di Rivalutazione}
Quando il Rispondente ha una mano sbilanciata e fit di almeno 4 carte, aggiunge ai suoi punti un valore dato dalla
differenza tra il numero di carte di atout possedute in mano e il numero di carte nel colore più corto.

\begin{bidtable}
    1\He/\Sp--\+\\
    1\Sp \> \> Naturale 4+\Sp, 5+ punti \\
    1\SA \> Naturale 5--10 senza altra dichiarazione \\
    2\Cl \>\> Naturale forzante manche, 2+\Cl\\
    2\Di \>\> 4+\Di, forzante manche\\
    2\He \>\> Appoggio 5--10 su 1\He, naturale forzante manche su 1\Sp\\
    2\Sp\>\> Non previsto su 1\He, appoggio 5--10 su 1\Sp\\
    2\SA \>\> Invitante\\
    3\Cl,\Di \>\> Invitante, 6+\Cl/\Di\\
    3\He \>\> Appoggio di 11 punti, anche rivalutati su 1\He; 6+\He invitanti su 1\Sp\\
    3\Sp \>\> Non previsto su 1\He; appoggio di 11 punti, anche rivalutati su 1\Sp\\
    3\SA \>\> 12--16 punti, bilanciato\\
    4\He/\Sp \>\> ``Giochiamo manche'', meno di 12 punti onori ma più di 12 punti rivalutati
\end{bidtable}
\newpage

\begin{bidtable}
    1\SA--\+\\
    2\Cl\>\> Stayman\+\+\\
    2\Di\> No 4e maggiori\\
    2\He \> 4\He, non 4\Sp\\
    2\Sp \> 4\Sp, non 4\He\\
    2\SA \> 4\He e 4\Sp, minimo\\
    3\Cl \> 4\He e 4\Sp, massimo\-\-\\
    2\Di,\He,\Sp \>\> 5+\Di/\He/\Sp, a passare\\
    2\SA \>\> Invitante, 8--9 punti\\
    3X \>\> Invitante naturale, 6+ carte\\
    3\SA \>\> A giocare\\
    4\He/\Sp \>\> A giocare
\end{bidtable}
Per mostrare 5+ carte in un maggiore in mano forzante manche, passare per 2\Cl e poi dichiarare il proprio maggiore.

\begin{bidtable}
    2\SA--\+\\
    3\Cl\>\> Stayman\+\+\\
    3\Di\> No 4e maggiori\\
    3\He \> 4\He, non 4\Sp\\
    3\Sp \> 4\Sp, non 4\He\\
    3\SA \> 4\He e 4\Sp\-\-\\
    3\Di,\He,\Sp \>\> 5+\Di/\He/\Sp, forzante\\
    3\SA \>\> A giocare\\
    4\He/\Sp \>\> A giocare
\end{bidtable}
L'unico parziale giocabile è 2\SA.

\section*{Quando interviene l'avversario}
Intervento di contro: colori 5\textsuperscript{i} non forzanti. Con tutte le mani di 11+ punti: Surcontro. Appoggi da sistema. \emph{No surlicita.}

Intervento a colore: nuovi colori 5\textsuperscript{i} forzanti; Contro 8+ punti, possibilità di giocare negli altri
colori, meno marcata mano a mano che il punteggio aumenta. \emph{No surlicita.}

\section*{Iterventi}
Interventi solidi. No salti. A livello 1: almeno un onore maggiore nel colore di intervento e 7+ punti (se pochi, concentrati nel
colore). Fino a 16 punti, altrimenti: Contro. \emph{No surlicita in risposta.}

A livello 2: 10--16 punti, bel colore, 6 carte tassative. Unica eccezione: su 1\Sp, la dichiarazione di 2\He può venire
da 5\He belle (almeno tre dei cinque onori) e 13+ punti. \emph{No surlicita in risposta.}

Contro: 12+ punti, giocabilità nei tre (due se dichiarato dopo due dichiarazioni avversarie) colori rimanenti. Su
1\Cl/\Di garantisce la 4--3 nobile. Inoltre, può contenere qualsiasi mano
di 17+ punti. In risposta, niente surlicita, dichiarare quello che si pensa di fare. Salti invitanti. \emph{No surlicita in risposta.}


\begin{landscape}

    % Drawing part, node distance is 1.5 cm and every node
    % is prefilled with white background
    \begin{tikzpicture}[node distance=2cm,
        every node/.style={font=\sffamily}, align=center]
        \tikzset{%
            >={Latex[width=2mm,length=2mm]},
            % Specifications for style of nodes:
            base/.style = {rectangle, rounded corners, draw=black,
                minimum width=4cm, minimum height=1cm,
            text centered, font=\sffamily},
            activityStarts/.style = {base, fill=blue!30},
            startstop/.style = {base, fill=red!30},
            activityRuns/.style = {base, fill=green!30},
            process/.style = {base, minimum width=2.5cm, fill=orange!15,
            font=\ttfamily},
        }
        % Specification of nodes (position, etc.)
        \node   (Aprire)    [activityStarts]    {12+ punti?};
        \node   (Passo)     [startstop, right of=Aprire, xshift=4cm]         {Passo};
        \node   (Bil)       [process, below of=Aprire]           {Bilanciato?};
        \node   (1sa?)       [process, below of=Bil]           {15--17 punti?};
        \node   (1sa)       [activityRuns, left of=1sa?, xshift=-4cm]      {1\SA};
        \node   (2sa?)       [process, below of=1sa?]           {21-22 punti?};
        \node   (2sa)       [activityRuns, below of=1sa]      {2\SA};
        \node   (23+?)       [process, below of=2sa?]           {23+ punti?};
        \node   (5p1)       [process, below of=23+?]           {5+\Sp?};
        \node   (1p)       [activityRuns, left of=5p1, xshift=-4cm]      {1\Sp};
        \node   (5h1)       [process, below of=5p1]           {5+\He?};
        \node   (1h)       [activityRuns, below of=1p]      {1\He};
        \node   (4d)       [process, below of=5h1]           {4+\Di?};
        \node   (1d)       [activityRuns, below of=1h]      {1\Di};
        \node   (1c)       [activityRuns, below of=4d]      {1\Cl};
        \node   (apri2)     [process, right of=Bil, xshift=4cm]           {21+ punti?};
        \node   (5p2)       [process, right of=5p1, xshift=4cm]           {5+\Sp?};
        \node   (2p)       [activityRuns, right of=5p2, xshift = 4cm]      {2\Sp};
        \node   (5h2)       [process, below of=5p2]           {5+\He?};
        \node   (2h)       [activityRuns, below of=2p]      {2\He};
        \node   (5d)       [process,below of=5h2]           {4+\Di?};
        \node   (2d)       [activityRuns, below of=2h]      {2\Di};
        \node   (2c)       [activityRuns, below of=5d]      {2\Cl};
        % Specification of lines between nodes specified above
        % with aditional nodes for description 
        \draw[-{>[color=green!80]}, color=green!80]             (Aprire) -- (Bil);
        \draw[-{>[color=green!80]}, color=green!80]     (Bil) -- (1sa?);
        \draw[-{>[color=green!80]}, color=green!80]      (1sa?) -- (1sa);
        \draw[-{>[color=green!80]}, color=green!80]      (apri2) -- (5p2);
        \draw[-{>[color=green!80]}, color=green!80]     (2sa?) -- (2sa);
        \draw[-{>[color=green!80]}, color=green!80]     (5p1) -- (1p);
        \draw[-{>[color=green!80]}, color=green!80]     (5h1) -- (1h);
        \draw[-{>[color=green!80]}, color=green!80]     (4d) -- (1d);
        \draw[-{>[color=green!80]}, color=green!80]     (5p2) -- (2p);
        \draw[-{>[color=green!80]}, color=green!80]     (5h2) -- (2h);
        \draw[-{>[color=green!80]}, color=green!80]     (5d) -- (2d);
        \draw[-{>[color=green!80]}, color=green!80]     (23+?.east) -- ++(1.6,0) -- ++(0,-8) --
            node[xshift=1.2cm,yshift=-1.5cm, text width=2.5cm]
            {} (2c.west);
        \draw[-{Circle[fill=red!80]}, color=red!80]             (Aprire) -- (Passo);
        \draw[-{Circle[fill=red!80]}, color=red!80]     (Bil) -- (apri2);
        \draw[-{Circle[fill=red!80]}, color=red!80]      (1sa?) -- (2sa?);
        \draw[-{Circle[fill=red!80]}, color=red!80]      (2sa?) -- (23+?);
        \draw[-{Circle[fill=red!80]}, color=red!80]      (23+?) -- (5p1);
        \draw[-{Circle[fill=red!80]}, color=red!80]      (5p1) -- (5h1);
        \draw[-{Circle[fill=red!80]}, color=red!80]      (5h1) -- (4d);
        \draw[-{Circle[fill=red!80]}, color=red!80]      (4d) -- (1c);
        \draw[-{Circle[fill=red!80]}, color=red!80]      (5p2) -- (5h2);
        \draw[-{Circle[fill=red!80]}, color=red!80]      (5h2) -- (5d);
        \draw[-{Circle[fill=red!80]}, color=red!80]      (5d) -- (2c);
        \draw[-{Circle[fill=red!80]}, color=red!80]      ([yshift=-.2cm]apri2.west) -- ++(-2.9,0) -- ++(0,-7.8) --(5p1.east);
    \end{tikzpicture}


\end{landscape}

    % Drawing part, node distance is 1.5 cm and every node
    % is prefilled with white background
    \begin{tikzpicture}[node distance=2cm,
        every node/.style={font=\sffamily}, align=center]
        \tikzset{%
            >={Latex[width=2mm,length=2mm]},
            % Specifications for style of nodes:
            base/.style = {rectangle, rounded corners, draw=black,
                minimum width=4cm, minimum height=1cm,
            text centered, font=\sffamily},
            activityStarts/.style = {base, fill=blue!30},
            startstop/.style = {base, fill=red!30},
            activityRuns/.style = {base, fill=green!30},
            process/.style = {base, minimum width=2.5cm, fill=orange!15,
            font=\ttfamily},
        }
        % Specification of nodes (position, etc.)
        \node   (Intervento)    [activityStarts]    {7+ punti?};
        \node   (Passo)     [startstop, right of=Intervento, xshift=4cm]         {Passo};
        \node   (1sa?)       [process, below of=Intervento]           {Requisiti per apertura 1\SA\\e fermo colore apertura?};
        \node   (1sa)       [activityRuns, left of=1sa?, xshift=-4cm]      {1\SA};
        \node   (Rever)       [process, below of=1sa?]           {17+ punti?};
        \node   (CRever)       [activityRuns, below of=1sa]      {Contro};
        \node   (5e?)       [process, below of=Rever, yshift=-.7cm]           {5(6)+ carte in un seme e\\requisiti per intervento\\a livello 1(2)?};
        \node   (1colore)       [activityRuns, below of=5e?, yshift=-.7cm]      {1\kern-.03em(2\kern.02em) nel colore};
        \node   (12+?)       [process, right of=5e?, xshift=4cm]           {12+?};
        \node   (doblabile?)       [process, right of=1colore, xshift = 4cm]      {Giocabilità nei tre semi\\diversi dall'apertura?};
        \node   (contro)       [activityRuns, below of=doblabile?]           {Contro};
        % Specification of lines between nodes specified above
        % with aditional nodes for description 
        \draw[-{>[color=green!80]}, color=green!80]             (Intervento) -- (1sa?);
        \draw[-{>[color=green!80]}, color=green!80]      (Rever) -- (CRever);
        \draw[-{>[color=green!80]}, color=green!80]      (1sa?) -- (1sa);
        \draw[-{>[color=green!80]}, color=green!80]      (5e?) -- (1colore);
        \draw[-{>[color=green!80]}, color=green!80]     (12+?) -- (doblabile?);
        \draw[-{>[color=green!80]}, color=green!80]     (doblabile?) -- (contro);
        \draw[-{Circle[fill=red!80]}, color=red!80]             (Intervento) -- (Passo);
        \draw[-{Circle[fill=red!80]}, color=red!80]      (12+?) -- (Passo);
        \draw[-{Circle[fill=red!80]}, color=red!80]      (Rever) -- (5e?);
        \draw[-{Circle[fill=red!80]}, color=red!80]      (5e?) -- (12+?);
        \draw[-{Circle[fill=red!80]}, color=red!80]      (12+?) -- (Passo);
        \draw[-{Circle[fill=red!80]}, color=red!80]      ([xshift=1.5cm]doblabile?.north) -- ([xshift=1.5cm]Passo.south);
        \draw[-{Circle[fill=red!80]}, color=red!80]      (1sa?) -- (Rever);
    \end{tikzpicture}
\end{document}
